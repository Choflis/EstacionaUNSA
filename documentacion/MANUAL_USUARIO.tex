% Options for packages loaded elsewhere
\PassOptionsToPackage{unicode}{hyperref}
\PassOptionsToPackage{hyphens}{url}
\documentclass[
]{article}
\usepackage{xcolor}
\usepackage{amsmath,amssymb}
\setcounter{secnumdepth}{5}
\usepackage{iftex}
\ifPDFTeX
  \usepackage[T1]{fontenc}
  \usepackage[utf8]{inputenc}
  \usepackage{textcomp} % provide euro and other symbols
\else % if luatex or xetex
  \usepackage{unicode-math} % this also loads fontspec
  \defaultfontfeatures{Scale=MatchLowercase}
  \defaultfontfeatures[\rmfamily]{Ligatures=TeX,Scale=1}
\fi
\usepackage{lmodern}
\ifPDFTeX\else
  % xetex/luatex font selection
\fi
% Use upquote if available, for straight quotes in verbatim environments
\IfFileExists{upquote.sty}{\usepackage{upquote}}{}
\IfFileExists{microtype.sty}{% use microtype if available
  \usepackage[]{microtype}
  \UseMicrotypeSet[protrusion]{basicmath} % disable protrusion for tt fonts
}{}
\makeatletter
\@ifundefined{KOMAClassName}{% if non-KOMA class
  \IfFileExists{parskip.sty}{%
    \usepackage{parskip}
  }{% else
    \setlength{\parindent}{0pt}
    \setlength{\parskip}{6pt plus 2pt minus 1pt}}
}{% if KOMA class
  \KOMAoptions{parskip=half}}
\makeatother
\usepackage{longtable,booktabs,array}
\usepackage{calc} % for calculating minipage widths
% Correct order of tables after \paragraph or \subparagraph
\usepackage{etoolbox}
\makeatletter
\patchcmd\longtable{\par}{\if@noskipsec\mbox{}\fi\par}{}{}
\makeatother
% Allow footnotes in longtable head/foot
\IfFileExists{footnotehyper.sty}{\usepackage{footnotehyper}}{\usepackage{footnote}}
\makesavenoteenv{longtable}
\usepackage{graphicx}
\makeatletter
\newsavebox\pandoc@box
\newcommand*\pandocbounded[1]{% scales image to fit in text height/width
  \sbox\pandoc@box{#1}%
  \Gscale@div\@tempa{\textheight}{\dimexpr\ht\pandoc@box+\dp\pandoc@box\relax}%
  \Gscale@div\@tempb{\linewidth}{\wd\pandoc@box}%
  \ifdim\@tempb\p@<\@tempa\p@\let\@tempa\@tempb\fi% select the smaller of both
  \ifdim\@tempa\p@<\p@\scalebox{\@tempa}{\usebox\pandoc@box}%
  \else\usebox{\pandoc@box}%
  \fi%
}
% Set default figure placement to htbp
\def\fps@figure{htbp}
\makeatother
\ifLuaTeX
\usepackage[bidi=basic]{babel}
\else
\usepackage[bidi=default]{babel}
\fi
\babelprovide[main,import]{spanish}
% get rid of language-specific shorthands (see #6817):
\let\LanguageShortHands\languageshorthands
\def\languageshorthands#1{}
\setlength{\emergencystretch}{3em} % prevent overfull lines
\providecommand{\tightlist}{%
  \setlength{\itemsep}{0pt}\setlength{\parskip}{0pt}}
\usepackage{bookmark}
\IfFileExists{xurl.sty}{\usepackage{xurl}}{} % add URL line breaks if available
\urlstyle{same}
\hypersetup{
  pdflang={es},
  hidelinks,
  pdfcreator={LaTeX via pandoc}}

\author{}
\date{}

\begin{document}

{
\setcounter{tocdepth}{3}
\tableofcontents
}
\section{ Manual de Usuario -
EstacionaUNSA}\label{manual-de-usuario---estacionaunsa}

\textbf{Sistema de Gestión de Estacionamientos Universitarios}

\begin{center}\rule{0.5\linewidth}{0.5pt}\end{center}

\textbf{Universidad Nacional de San Agustín de Arequipa}\\
\textbf{Escuela Profesional de Ingeniería de Sistemas}\\
\textbf{Curso:} Construcción de Software

\textbf{Equipo de Desarrollo:} - Luis Guillermo Luque Condori - Dennis
Javier Quispe Saavedra - Fernando Miguel Garambel Marín

\textbf{Versión:} 1.0\\
\textbf{Fecha:} Diciembre 2025

\begin{center}\rule{0.5\linewidth}{0.5pt}\end{center}

\subsection{ Índice}\label{uxedndice}

\begin{enumerate}
\def\labelenumi{\arabic{enumi}.}
\item
  \hyperref[1-introducciuxf3n]{Introducción}

  \begin{itemize}
  \tightlist
  \item
    1.1 \hyperref[11-quuxe9-es-estacionaunsa]{¿Qué es EstacionaUNSA?}
  \item
    1.2 \hyperref[12-caracteruxedsticas-principales]{Características
    Principales}
  \item
    1.3 \hyperref[13-requisitos-del-sistema]{Requisitos del Sistema}
  \item
    1.4 \hyperref[14-instalaciuxf3n-de-la-aplicaciuxf3n]{Instalación de
    la Aplicación}
  \end{itemize}
\item
  \hyperref[2-primeros-pasos]{Primeros Pasos}

  \begin{itemize}
  \tightlist
  \item
    2.1 \hyperref[21-registro-de-cuenta]{Registro de Cuenta}
  \item
    2.2 \hyperref[22-inicio-de-sesiuxf3n]{Inicio de Sesión}
  \item
    2.3 \hyperref[23-recuperaciuxf3n-de-contraseuxf1a]{Recuperación de
    Contraseña}
  \end{itemize}
\item
  \hyperref[3-gestiuxf3n-de-perfil]{Gestión de Perfil}

  \begin{itemize}
  \tightlist
  \item
    3.1 \hyperref[31-ver-y-editar-perfil]{Ver y Editar Perfil}
  \item
    3.2 \hyperref[32-agregar-vehuxedculos]{Agregar Vehículos}
  \item
    3.3 \hyperref[33-editar-y-eliminar-vehuxedculos]{Editar y Eliminar
    Vehículos}
  \end{itemize}
\item
  \hyperref[4-sistema-de-estacionamiento]{Sistema de Estacionamiento}

  \begin{itemize}
  \tightlist
  \item
    4.1 \hyperref[41-visualizar-zonas-disponibles]{Visualizar Zonas
    Disponibles}
  \item
    4.2 \hyperref[42-mapa-de-estacionamiento]{Mapa de Estacionamiento}
  \item
    4.3 \hyperref[43-estados-de-los-espacios]{Estados de los Espacios}
  \end{itemize}
\item
  \hyperref[5-realizar-reservas]{Realizar Reservas}

  \begin{itemize}
  \tightlist
  \item
    5.1 \hyperref[51-cuxf3mo-hacer-una-reserva]{Cómo Hacer una Reserva}
  \item
    5.2 \hyperref[52-restricciones-de-reserva]{Restricciones de Reserva}
  \item
    5.3 \hyperref[53-cancelar-una-reserva]{Cancelar una Reserva}
  \item
    5.4 \hyperref[54-quuxe9-hacer-al-llegar]{Qué Hacer al Llegar}
  \end{itemize}
\item
  \hyperref[6-historial-y-notificaciones]{Historial y Notificaciones}

  \begin{itemize}
  \tightlist
  \item
    6.1 \hyperref[61-ver-historial-de-uso]{Ver Historial de Uso}
  \item
    6.2 \hyperref[62-gestionar-notificaciones]{Gestionar Notificaciones}
  \item
    6.3 \hyperref[63-sistema-de-penalizaciones]{Sistema de
    Penalizaciones}
  \end{itemize}
\item
  \hyperref[7-preguntas-frecuentes-faq]{Preguntas Frecuentes (FAQ)}
\item
  \hyperref[8-soluciuxf3n-de-problemas]{Solución de Problemas}
\item
  \hyperref[9-contacto-y-soporte]{Contacto y Soporte}
\end{enumerate}

\begin{center}\rule{0.5\linewidth}{0.5pt}\end{center}

\subsection{1. Introducción}\label{introducciuxf3n}

\subsubsection{1.1 ¿Qué es
EstacionaUNSA?}\label{quuxe9-es-estacionaunsa}

EstacionaUNSA es una aplicación móvil diseñada para optimizar la gestión
de estacionamientos en la Universidad Nacional de San Agustín de
Arequipa (UNSA). La aplicación permite a estudiantes, docentes y
personal administrativo visualizar la disponibilidad de espacios de
estacionamiento en tiempo real, realizar reservas y consultar su
historial de uso.

\textbf{Beneficios principales:} -  Ahorro de tiempo al buscar
estacionamiento -  Visualización en tiempo real de espacios
disponibles -  Sistema de reservas para garantizar tu espacio - 
Notificaciones automáticas -  Control de acceso institucional

\begin{figure}
\centering
\pandocbounded{\includegraphics[keepaspectratio]{documentacion/imagenes/manual_usuario/01_pantalla_principal.png}}
\caption{Pantalla principal de EstacionaUNSA}
\end{figure}

\begin{center}\rule{0.5\linewidth}{0.5pt}\end{center}

\subsubsection{1.2 Características
Principales}\label{caracteruxedsticas-principales}

EstacionaUNSA ofrece las siguientes funcionalidades:

\begin{longtable}[]{@{}
  >{\raggedright\arraybackslash}p{(\linewidth - 2\tabcolsep) * \real{0.5517}}
  >{\raggedright\arraybackslash}p{(\linewidth - 2\tabcolsep) * \real{0.4483}}@{}}
\toprule\noalign{}
\begin{minipage}[b]{\linewidth}\raggedright
Característica
\end{minipage} & \begin{minipage}[b]{\linewidth}\raggedright
Descripción
\end{minipage} \\
\midrule\noalign{}
\endhead
\bottomrule\noalign{}
\endlastfoot
\textbf{Autenticación Segura} & Acceso exclusivo con correos
institucionales @unsa.edu.pe \\
\textbf{Gestión de Vehículos} & Registra y administra múltiples
vehículos (autos y motocicletas) \\
\textbf{Visualización en Tiempo Real} & Consulta la disponibilidad de
espacios al instante \\
\textbf{Sistema de Reservas} & Reserva un espacio por hasta 15
minutos \\
\textbf{Mapa Interactivo} & Visualiza las zonas de estacionamiento en un
mapa \\
\textbf{Historial Completo} & Revisa todas tus entradas y salidas \\
\textbf{Notificaciones Push} & Recibe alertas sobre tus reservas y
espacios disponibles \\
\textbf{Sistema de Penalizaciones} & Control automático de no-shows para
uso justo \\
\end{longtable}

\begin{center}\rule{0.5\linewidth}{0.5pt}\end{center}

\subsubsection{1.3 Requisitos del Sistema}\label{requisitos-del-sistema}

\paragraph{Requisitos Mínimos}\label{requisitos-muxednimos}

\textbf{Para Android:} - Sistema operativo: Android 6.0 (Marshmallow) o
superior - Espacio de almacenamiento: 50 MB libres - Conexión a
Internet: WiFi o datos móviles - GPS activado (para funciones de
ubicación)

\textbf{Para iOS:} - Sistema operativo: iOS 11.0 o superior - Espacio de
almacenamiento: 50 MB libres - Conexión a Internet: WiFi o datos móviles
- Servicios de ubicación activados

\paragraph{Requisitos de Cuenta}\label{requisitos-de-cuenta}

\begin{itemize}
\tightlist
\item
  Correo electrónico institucional válido (@unsa.edu.pe)
\item
  Ser miembro activo de la comunidad UNSA (estudiante, docente o
  personal administrativo)
\item
  Tener al menos un vehículo registrado para usar el sistema de reservas
\end{itemize}

\begin{center}\rule{0.5\linewidth}{0.5pt}\end{center}

\subsubsection{1.4 Instalación de la
Aplicación}\label{instalaciuxf3n-de-la-aplicaciuxf3n}

\paragraph{Instalación en Android}\label{instalaciuxf3n-en-android}

\begin{enumerate}
\def\labelenumi{\arabic{enumi}.}
\tightlist
\item
  \textbf{Descarga el archivo APK:}

  \begin{itemize}
  \tightlist
  \item
    Accede al enlace proporcionado por la universidad
  \item
    Descarga el archivo \texttt{estaciona-unsa.apk}
  \end{itemize}
\item
  \textbf{Habilita instalación de fuentes desconocidas:}

  \begin{itemize}
  \tightlist
  \item
    Ve a \textbf{Configuración} \textgreater{} \textbf{Seguridad}
  \item
    Activa \textbf{Fuentes desconocidas} o \textbf{Instalar aplicaciones
    desconocidas}
  \end{itemize}
\item
  \textbf{Instala la aplicación:}

  \begin{itemize}
  \tightlist
  \item
    Abre el archivo APK descargado
  \item
    Toca \textbf{Instalar}
  \item
    Espera a que finalice la instalación
  \item
    Toca \textbf{Abrir} para iniciar la aplicación
  \end{itemize}
\item
  \textbf{Concede permisos necesarios:}

  \begin{itemize}
  \tightlist
  \item
    Ubicación (para detectar proximidad a las zonas)
  \item
    Notificaciones (para recibir alertas)
  \item
    Almacenamiento (para guardar fotos de vehículos)
  \end{itemize}
\end{enumerate}

\begin{figure}
\centering
\pandocbounded{\includegraphics[keepaspectratio]{documentacion/imagenes/manual_usuario/02_instalacion_android.png}}
\caption{Instalación en Android}
\end{figure}

\paragraph{Instalación en iOS}\label{instalaciuxf3n-en-ios}

La aplicación estará disponible próximamente en la App Store. Por ahora,
la versión está disponible solo para Android.

\begin{center}\rule{0.5\linewidth}{0.5pt}\end{center}

\subsection{2. Primeros Pasos}\label{primeros-pasos}

\subsubsection{2.1 Registro de Cuenta}\label{registro-de-cuenta}

Para usar EstacionaUNSA, primero debes crear una cuenta con tu correo
institucional.

\textbf{Pasos para Ingresar:}

\begin{enumerate}
\def\labelenumi{\arabic{enumi}.}
\item
  \textbf{Abre la aplicación} EstacionaUNSA en tu dispositivo
\item
  \textbf{En la pantalla de inicio}, toca el botón \textbf{``Iniciar
  cuenta con Google''}
\item
  \textbf{¡Listo!} Tu cuenta ha sido creada exitosamente
\end{enumerate}

\begin{quote}
\textbf{ Importante:} Solo se permiten correos con dominio
@unsa.edu.pe. Si intentas iniciar sesión con otro correo, el sistema
mostrará un error.
\end{quote}

\begin{figure}
\centering
\pandocbounded{\includegraphics[keepaspectratio]{documentacion/imagenes/manual_usuario/03_registro_cuenta.png}}
\caption{Registro de cuenta}
\end{figure}

\begin{center}\rule{0.5\linewidth}{0.5pt}\end{center}

\subsubsection{2.2 Inicio de Sesión}\label{inicio-de-sesiuxf3n}

Una vez que hayas creado tu cuenta, puedes iniciar sesión en cualquier
momento.

\textbf{Pasos para iniciar sesión:}

\begin{enumerate}
\def\labelenumi{\arabic{enumi}.}
\item
  \textbf{Abre la aplicación} EstacionaUNSA
\item
  \textbf{En la pantalla de inicio}, ingresa:

  \begin{itemize}
  \tightlist
  \item
    \textbf{Correo electrónico:} Tu correo institucional
  \item
    \textbf{Contraseña:} La contraseña que creaste
  \end{itemize}
\item
  \textbf{Toca el botón ``Iniciar Sesión''}
\item
  \textbf{La aplicación te llevará a la pantalla principal} donde podrás
  ver las zonas de estacionamiento disponibles
\end{enumerate}

\textbf{Opciones adicionales:} -  \textbf{Recordar sesión:} Marca esta
opción para no tener que iniciar sesión cada vez -  \textbf{Modo
seguro:} Si usas un dispositivo compartido, no marques ``Recordar
sesión''

\begin{figure}
\centering
\pandocbounded{\includegraphics[keepaspectratio]{documentacion/imagenes/manual_usuario/04_inicio_sesion.png}}
\caption{Inicio de sesión}
\end{figure}

\begin{center}\rule{0.5\linewidth}{0.5pt}\end{center}

\subsubsection{2.3 Recuperación de
Contraseña}\label{recuperaciuxf3n-de-contraseuxf1a}

Si olvidaste tu contraseña, puedes recuperarla fácilmente.

\textbf{Pasos para recuperar tu contraseña:}

\begin{enumerate}
\def\labelenumi{\arabic{enumi}.}
\item
  \textbf{En la pantalla de inicio de sesión}, toca \textbf{``¿Olvidaste
  tu contraseña?''}
\item
  \textbf{Ingresa tu correo electrónico} institucional
\item
  \textbf{Toca ``Enviar enlace de recuperación''}
\item
  \textbf{Revisa tu correo electrónico:}

  \begin{itemize}
  \tightlist
  \item
    Abre el correo de recuperación
  \item
    Haz clic en el enlace proporcionado
  \end{itemize}
\item
  \textbf{Crea una nueva contraseña:}

  \begin{itemize}
  \tightlist
  \item
    Ingresa tu nueva contraseña
  \item
    Confírmala
  \item
    Guarda los cambios
  \end{itemize}
\item
  \textbf{Inicia sesión} con tu nueva contraseña
\end{enumerate}

\begin{quote}
\textbf{ Consejo:} Usa una contraseña segura que incluya letras
mayúsculas, minúsculas, números y símbolos.
\end{quote}

\begin{center}\rule{0.5\linewidth}{0.5pt}\end{center}

\subsection{3. Gestión de Perfil}\label{gestiuxf3n-de-perfil}

\subsubsection{3.1 Ver y Editar Perfil}\label{ver-y-editar-perfil}

Tu perfil contiene tu información personal y estadísticas de uso del
sistema.

\textbf{Acceder a tu perfil:}

\begin{enumerate}
\def\labelenumi{\arabic{enumi}.}
\item
  \textbf{Desde la pantalla principal}, toca el ícono de \textbf{perfil}
  () en la barra de navegación inferior
\item
  \textbf{Visualiza tu información:}

  \begin{itemize}
  \tightlist
  \item
    Foto de perfil
  \item
    Nombre completo
  \item
    Correo electrónico
  \item
    Estadísticas de uso (reservas totales, completadas, etc.)
  \item
    Lista de vehículos registrados
  \end{itemize}
\end{enumerate}

\textbf{Editar tu perfil:}

\begin{enumerate}
\def\labelenumi{\arabic{enumi}.}
\item
  \textbf{En la pantalla de perfil}, toca el botón \textbf{``Editar
  perfil''} o el ícono de lápiz ()
\item
  \textbf{Modifica la información que desees:}

  \begin{itemize}
  \tightlist
  \item
    Cambiar foto de perfil (toca la foto actual)
  \item
    Actualizar nombre
  \item
    Cambiar contraseña
  \end{itemize}
\item
  \textbf{Toca ``Guardar cambios''} para aplicar las modificaciones
\end{enumerate}

\begin{figure}
\centering
\pandocbounded{\includegraphics[keepaspectratio]{documentacion/imagenes/manual_usuario/05_perfil_usuario.png}}
\caption{Perfil de usuario}
\end{figure}

\begin{center}\rule{0.5\linewidth}{0.5pt}\end{center}

\subsubsection{3.2 Agregar Vehículos}\label{agregar-vehuxedculos}

Para poder hacer reservas, necesitas tener al menos un vehículo
registrado en tu cuenta.

\textbf{Pasos para agregar un vehículo:}

\begin{enumerate}
\def\labelenumi{\arabic{enumi}.}
\item
  \textbf{Ve a tu perfil} tocando el ícono de perfil ()
\item
  \textbf{En la sección ``Mis Vehículos''}, toca el botón
  \textbf{``Agregar vehículo''} o el ícono de más (+)
\item
  \textbf{Completa la información del vehículo:}

  \begin{itemize}
  \tightlist
  \item
    \textbf{Placa:} Número de matrícula del vehículo (obligatorio)
  \item
    \textbf{Tipo:} Selecciona Auto o Motocicleta
  \item
    \textbf{Modelo:} Marca y modelo del vehículo (opcional)
  \item
    \textbf{Color:} Color del vehículo (opcional)
  \item
    \textbf{Foto:} Toma o selecciona una foto del vehículo (opcional
    pero recomendado)
  \end{itemize}
\item
  \textbf{Toca ``Guardar vehículo''}
\item
  \textbf{El vehículo aparecerá en tu lista} de vehículos registrados
\end{enumerate}

\begin{quote}
\textbf{ Importante:} La placa debe ser válida y estar asociada a un
vehículo real. El personal de vigilancia verificará esta información al
momento de tu ingreso.
\end{quote}

\begin{figure}
\centering
\pandocbounded{\includegraphics[keepaspectratio]{documentacion/imagenes/manual_usuario/06_agregar_vehiculo.png}}
\caption{Agregar vehículo}
\end{figure}

\begin{center}\rule{0.5\linewidth}{0.5pt}\end{center}

\subsubsection{3.3 Editar y Eliminar
Vehículos}\label{editar-y-eliminar-vehuxedculos}

Puedes modificar o eliminar vehículos registrados en cualquier momento.

\textbf{Editar un vehículo:}

\begin{enumerate}
\def\labelenumi{\arabic{enumi}.}
\item
  \textbf{En la sección ``Mis Vehículos''} de tu perfil, toca el
  vehículo que deseas editar
\item
  \textbf{Modifica la información} que necesites cambiar
\item
  \textbf{Toca ``Guardar cambios''}
\end{enumerate}

\textbf{Eliminar un vehículo:}

\begin{enumerate}
\def\labelenumi{\arabic{enumi}.}
\item
  \textbf{En la lista de vehículos}, desliza el vehículo hacia la
  izquierda o toca el ícono de opciones ()
\item
  \textbf{Selecciona ``Eliminar''}
\item
  \textbf{Confirma la eliminación} en el diálogo que aparece
\end{enumerate}

\begin{quote}
\textbf{ Nota:} No puedes eliminar un vehículo si tiene una reserva
activa. Primero debes cancelar la reserva.
\end{quote}

\begin{center}\rule{0.5\linewidth}{0.5pt}\end{center}

\subsection{4. Sistema de
Estacionamiento}\label{sistema-de-estacionamiento}

\subsubsection{4.1 Visualizar Zonas
Disponibles}\label{visualizar-zonas-disponibles}

EstacionaUNSA gestiona tres zonas principales de estacionamiento en el
campus de la UNSA.

\textbf{Las tres zonas son:}

\begin{longtable}[]{@{}
  >{\raggedright\arraybackslash}p{(\linewidth - 6\tabcolsep) * \real{0.1667}}
  >{\raggedright\arraybackslash}p{(\linewidth - 6\tabcolsep) * \real{0.2222}}
  >{\raggedright\arraybackslash}p{(\linewidth - 6\tabcolsep) * \real{0.3056}}
  >{\raggedright\arraybackslash}p{(\linewidth - 6\tabcolsep) * \real{0.3056}}@{}}
\toprule\noalign{}
\begin{minipage}[b]{\linewidth}\raggedright
Zona
\end{minipage} & \begin{minipage}[b]{\linewidth}\raggedright
Nombre
\end{minipage} & \begin{minipage}[b]{\linewidth}\raggedright
Ubicación
\end{minipage} & \begin{minipage}[b]{\linewidth}\raggedright
Capacidad
\end{minipage} \\
\midrule\noalign{}
\endhead
\bottomrule\noalign{}
\endlastfoot
\textbf{Zona A} & Entrada Principal & Puerta principal de la UNSA & 50
espacios \\
\textbf{Zona B} & Biblioteca Central & Junto a la biblioteca & 30
espacios \\
\textbf{Zona C} & Ingenierías & Facultad de Ingeniería & 40 espacios \\
\end{longtable}

\textbf{Ver zonas disponibles:}

\begin{enumerate}
\def\labelenumi{\arabic{enumi}.}
\tightlist
\item
  \textbf{En la pantalla principal}, verás tarjetas con información de
  cada zona:

  \begin{itemize}
  \tightlist
  \item
    Nombre de la zona
  \item
    Espacios disponibles / Total de espacios
  \item
    Distancia desde tu ubicación actual
  \item
    Estado (Abierta / Cerrada)
  \end{itemize}
\item
  \textbf{Los colores indican disponibilidad:}

  \begin{itemize}
  \tightlist
  \item
     \textbf{Verde:} Muchos espacios disponibles (\textgreater30\%)
  \item
     \textbf{Amarillo:} Disponibilidad media (10-30\%)
  \item
     \textbf{Rojo:} Pocos espacios disponibles (\textless10\%)
  \item
     \textbf{Gris:} Zona cerrada o sin espacios
  \end{itemize}
\item
  \textbf{Toca una zona} para ver más detalles y espacios específicos
\end{enumerate}

\begin{figure}
\centering
\pandocbounded{\includegraphics[keepaspectratio]{documentacion/imagenes/manual_usuario/07_zonas_disponibles.png}}
\caption{Zonas disponibles}
\end{figure}

\begin{center}\rule{0.5\linewidth}{0.5pt}\end{center}

\subsubsection{4.2 Mapa de
Estacionamiento}\label{mapa-de-estacionamiento}

El mapa interactivo te permite visualizar la ubicación exacta de cada
zona y tu distancia a ellas.

\textbf{Usar el mapa:}

\begin{enumerate}
\def\labelenumi{\arabic{enumi}.}
\item
  \textbf{Desde la pantalla principal}, toca el ícono de \textbf{mapa}
  () en la barra de navegación
\item
  \textbf{El mapa mostrará:}

  \begin{itemize}
  \tightlist
  \item
    Tu ubicación actual (punto azul)
  \item
    Las tres zonas de estacionamiento (marcadores)
  \item
    Círculo de 500m de radio (área de reserva permitida)
  \end{itemize}
\item
  \textbf{Interactúa con el mapa:}

  \begin{itemize}
  \tightlist
  \item
    \textbf{Zoom:} Pellizca para acercar o alejar
  \item
    \textbf{Desplazar:} Arrastra para mover el mapa
  \item
    \textbf{Toca un marcador:} Ver información de la zona
  \end{itemize}
\item
  \textbf{Toca ``Ver detalles''} en la información de la zona para
  acceder a los espacios disponibles
\end{enumerate}

\begin{quote}
\textbf{ Consejo:} El mapa se actualiza automáticamente con tu
ubicación. Asegúrate de tener el GPS activado.
\end{quote}

\begin{figure}
\centering
\pandocbounded{\includegraphics[keepaspectratio]{documentacion/imagenes/manual_usuario/08_mapa_estacionamiento.png}}
\caption{Mapa de estacionamiento}
\end{figure}

\begin{center}\rule{0.5\linewidth}{0.5pt}\end{center}

\subsubsection{4.3 Estados de los
Espacios}\label{estados-de-los-espacios}

Cada espacio de estacionamiento puede tener uno de los siguientes
estados:

\begin{longtable}[]{@{}
  >{\raggedright\arraybackslash}p{(\linewidth - 6\tabcolsep) * \real{0.1702}}
  >{\raggedright\arraybackslash}p{(\linewidth - 6\tabcolsep) * \real{0.1489}}
  >{\raggedright\arraybackslash}p{(\linewidth - 6\tabcolsep) * \real{0.2766}}
  >{\raggedright\arraybackslash}p{(\linewidth - 6\tabcolsep) * \real{0.4043}}@{}}
\toprule\noalign{}
\begin{minipage}[b]{\linewidth}\raggedright
Estado
\end{minipage} & \begin{minipage}[b]{\linewidth}\raggedright
Color
\end{minipage} & \begin{minipage}[b]{\linewidth}\raggedright
Descripción
\end{minipage} & \begin{minipage}[b]{\linewidth}\raggedright
Acción Disponible
\end{minipage} \\
\midrule\noalign{}
\endhead
\bottomrule\noalign{}
\endlastfoot
\textbf{Disponible} &  Verde & El espacio está libre y puede ser
reservado & Reservar \\
\textbf{Reservado} &  Amarillo & Alguien ha reservado este espacio &
No disponible \\
\textbf{Ocupado} &  Rojo & Hay un vehículo estacionado actualmente &
No disponible \\
\textbf{Inactivo} &  Gris & El espacio está fuera de servicio & No
disponible \\
\end{longtable}

\textbf{Ver espacios de una zona:}

\begin{enumerate}
\def\labelenumi{\arabic{enumi}.}
\item
  \textbf{Toca una zona} desde la pantalla principal o el mapa
\item
  \textbf{Verás una cuadrícula o lista} con todos los espacios de esa
  zona
\item
  \textbf{Cada espacio muestra:}

  \begin{itemize}
  \tightlist
  \item
    Número del espacio (ej: A-001, B-015)
  \item
    Estado actual (color)
  \item
    Tipo (Auto / Motocicleta)
  \item
    Características especiales (techado, para discapacitados, etc.)
  \end{itemize}
\item
  \textbf{Toca un espacio disponible} para ver la opción de reservar
\end{enumerate}

\begin{figure}
\centering
\pandocbounded{\includegraphics[keepaspectratio]{documentacion/imagenes/manual_usuario/09_estados_espacios.png}}
\caption{Estados de los espacios}
\end{figure}

\begin{center}\rule{0.5\linewidth}{0.5pt}\end{center}

\subsection{5. Realizar Reservas}\label{realizar-reservas}

\subsubsection{5.1 Cómo Hacer una
Reserva}\label{cuxf3mo-hacer-una-reserva}

El sistema de reservas te permite asegurar un espacio antes de llegar a
la universidad.

\textbf{Pasos para hacer una reserva:}

\begin{enumerate}
\def\labelenumi{\arabic{enumi}.}
\item
  \textbf{Asegúrate de estar dentro del rango permitido:}

  \begin{itemize}
  \tightlist
  \item
    Debes estar a 500 metros o menos de la zona que deseas reservar
  \item
    La aplicación mostrará tu distancia actual
  \end{itemize}
\item
  \textbf{Selecciona una zona} desde la pantalla principal
\item
  \textbf{Elige un espacio disponible} (marcado en verde)
\item
  \textbf{Toca el botón ``Reservar''}
\item
  \textbf{Confirma tu reserva:}

  \begin{itemize}
  \tightlist
  \item
    Verifica el espacio seleccionado
  \item
    Selecciona el vehículo que usarás
  \item
    Lee las condiciones de la reserva
  \item
    Toca ``Confirmar reserva''
  \end{itemize}
\item
  \textbf{¡Reserva confirmada!}

  \begin{itemize}
  \tightlist
  \item
    Recibirás una notificación de confirmación
  \item
    El espacio se marcará como reservado
  \item
    Tienes \textbf{15 minutos} para llegar
  \end{itemize}
\item
  \textbf{Dirígete al espacio reservado:}

  \begin{itemize}
  \tightlist
  \item
    El vigilante verificará tu reserva
  \item
    Muestra la aplicación con tu reserva activa
  \item
    Estaciona en el espacio asignado
  \end{itemize}
\end{enumerate}

\begin{quote}
\textbf{ Importante:} Tienes exactamente 15 minutos desde que haces la
reserva para llegar. Si no llegas a tiempo, la reserva expirará y se
contará como ``no-show''.
\end{quote}

\begin{figure}
\centering
\pandocbounded{\includegraphics[keepaspectratio]{documentacion/imagenes/manual_usuario/10_hacer_reserva.png}}
\caption{Proceso de reserva}
\end{figure}

\begin{center}\rule{0.5\linewidth}{0.5pt}\end{center}

\subsubsection{5.2 Restricciones de
Reserva}\label{restricciones-de-reserva}

Para garantizar un uso justo del sistema, existen las siguientes
restricciones:

\paragraph{Restricción de Distancia}\label{restricciuxf3n-de-distancia}

\begin{itemize}
\tightlist
\item
   \textbf{Permitido:} Estar a 500 metros o menos de la zona
\item
  ❌ \textbf{No permitido:} Reservar desde casa o lugares lejanos
\end{itemize}

\textbf{¿Por qué?} Para evitar reservas anticipadas que bloqueen
espacios innecesariamente.

\paragraph{Restricción de Tiempo}\label{restricciuxf3n-de-tiempo}

\begin{itemize}
\tightlist
\item
   \textbf{Duración máxima:} 15 minutos
\item
   \textbf{No renovable:} No puedes extender una reserva activa
\end{itemize}

\textbf{¿Por qué?} Para mantener la rotación de espacios y dar
oportunidad a todos.

\paragraph{Restricción de Cantidad}\label{restricciuxf3n-de-cantidad}

\begin{itemize}
\tightlist
\item
  1️⃣ \textbf{Máximo:} 1 reserva activa por usuario
\item
  ❌ \textbf{No permitido:} Reservar múltiples espacios simultáneamente
\end{itemize}

\textbf{¿Por qué?} Para evitar acaparamiento de espacios.

\paragraph{Restricción de Horario}\label{restricciuxf3n-de-horario}

\begin{itemize}
\tightlist
\item
  🕐 \textbf{Horario de operación:} Lunes a Viernes, 6:00 AM - 10:00 PM
\item
  🕐 \textbf{Sábados:} 6:00 AM - 2:00 PM
\item
  ❌ \textbf{Domingos y feriados:} Cerrado
\end{itemize}

\paragraph{Restricción por
Penalizaciones}\label{restricciuxf3n-por-penalizaciones}

\begin{itemize}
\tightlist
\item
   \textbf{3 no-shows:} Advertencia
\item
  🚫 \textbf{5 no-shows:} Suspensión de 7 días
\item
  🔒 \textbf{10 no-shows:} Suspensión permanente
\end{itemize}

\begin{quote}
\textbf{ Consejo:} Sé responsable con tus reservas. Solo reserva
cuando estés seguro de que llegarás a tiempo.
\end{quote}

\begin{center}\rule{0.5\linewidth}{0.5pt}\end{center}

\subsubsection{5.3 Cancelar una Reserva}\label{cancelar-una-reserva}

Si por alguna razón no puedes llegar, es importante que canceles tu
reserva para liberar el espacio.

\textbf{Pasos para cancelar una reserva:}

\begin{enumerate}
\def\labelenumi{\arabic{enumi}.}
\item
  \textbf{Ve a la sección ``Mis Reservas''} tocando el ícono de reservas
  () en la barra de navegación
\item
  \textbf{Verás tu reserva activa} con:

  \begin{itemize}
  \tightlist
  \item
    Espacio reservado
  \item
    Tiempo restante
  \item
    Zona y ubicación
  \end{itemize}
\item
  \textbf{Toca el botón ``Cancelar reserva''}
\item
  \textbf{Confirma la cancelación} en el diálogo que aparece
\item
  \textbf{La reserva se cancelará inmediatamente:}

  \begin{itemize}
  \tightlist
  \item
    El espacio quedará disponible para otros
  \item
    No se contará como no-show
  \item
    Recibirás una notificación de confirmación
  \end{itemize}
\end{enumerate}

\begin{quote}
\textbf{ Buena práctica:} Si sabes que no llegarás, cancela tu reserva
lo antes posible. Esto ayuda a otros usuarios y evita penalizaciones en
tu cuenta.
\end{quote}

\begin{figure}
\centering
\pandocbounded{\includegraphics[keepaspectratio]{documentacion/imagenes/manual_usuario/11_cancelar_reserva.png}}
\caption{Cancelar reserva}
\end{figure}

\begin{center}\rule{0.5\linewidth}{0.5pt}\end{center}

\subsubsection{5.4 Qué Hacer al Llegar}\label{quuxe9-hacer-al-llegar}

Una vez que llegues a la universidad con una reserva activa:

\textbf{Proceso de entrada:}

\begin{enumerate}
\def\labelenumi{\arabic{enumi}.}
\item
  \textbf{Dirígete a la zona reservada} (Zona A, B o C)
\item
  \textbf{Muestra tu reserva al vigilante:}

  \begin{itemize}
  \tightlist
  \item
    Abre la aplicación
  \item
    Ve a ``Mis Reservas''
  \item
    Muestra la pantalla con tu reserva activa
  \end{itemize}
\item
  \textbf{El vigilante verificará:}

  \begin{itemize}
  \tightlist
  \item
    Tu identidad (CUI o carnet universitario)
  \item
    La placa de tu vehículo coincide con la registrada
  \item
    La reserva está activa y no ha expirado
  \end{itemize}
\item
  \textbf{Estaciona en el espacio asignado:}

  \begin{itemize}
  \tightlist
  \item
    Busca el número de espacio (ej: A-015)
  \item
    Estaciona correctamente dentro de las líneas
  \end{itemize}
\item
  \textbf{El vigilante registrará tu entrada:}

  \begin{itemize}
  \tightlist
  \item
    Tu reserva cambiará a estado ``Completada''
  \item
    El espacio se marcará como ``Ocupado''
  \item
    Se iniciará el registro de tu tiempo de estacionamiento
  \end{itemize}
\end{enumerate}

\textbf{Proceso de salida:}

\begin{enumerate}
\def\labelenumi{\arabic{enumi}.}
\item
  \textbf{Cuando vayas a salir}, dirígete a la caseta de vigilancia
\item
  \textbf{El vigilante registrará tu salida:}

  \begin{itemize}
  \tightlist
  \item
    Verificará tu placa
  \item
    Registrará la hora de salida
  \item
    Liberará el espacio
  \end{itemize}
\item
  \textbf{El espacio quedará disponible} para otros usuarios
\end{enumerate}

\begin{quote}
\textbf{ Importante:} Siempre estaciona en el espacio exacto que
reservaste. Estacionar en otro espacio puede resultar en una incidencia.
\end{quote}

\begin{center}\rule{0.5\linewidth}{0.5pt}\end{center}

\subsection{6. Historial y
Notificaciones}\label{historial-y-notificaciones}

\subsubsection{6.1 Ver Historial de Uso}\label{ver-historial-de-uso}

El historial te permite revisar todas tus reservas y usos del
estacionamiento.

\textbf{Acceder al historial:}

\begin{enumerate}
\def\labelenumi{\arabic{enumi}.}
\item
  \textbf{Toca el ícono de historial} () en la barra de navegación
\item
  \textbf{Verás una lista de todas tus reservas:}

  \begin{itemize}
  \tightlist
  \item
    Reservas completadas
  \item
    Reservas canceladas
  \item
    Reservas expiradas (no-shows)
  \end{itemize}
\item
  \textbf{Cada entrada muestra:}

  \begin{itemize}
  \tightlist
  \item
    Fecha y hora
  \item
    Zona y espacio utilizado
  \item
    Duración del estacionamiento
  \item
    Estado final (Completada, Cancelada, Expirada)
  \end{itemize}
\item
  \textbf{Toca una entrada} para ver detalles completos:

  \begin{itemize}
  \tightlist
  \item
    Hora de reserva
  \item
    Hora de entrada (si aplica)
  \item
    Hora de salida (si aplica)
  \item
    Tiempo total estacionado
  \item
    Vehículo utilizado
  \end{itemize}
\end{enumerate}

\textbf{Filtrar el historial:}

\begin{itemize}
\tightlist
\item
  \textbf{Por fecha:} Selecciona un rango de fechas
\item
  \textbf{Por estado:} Filtra por completadas, canceladas o expiradas
\item
  \textbf{Por zona:} Muestra solo una zona específica
\end{itemize}

\begin{figure}
\centering
\pandocbounded{\includegraphics[keepaspectratio]{documentacion/imagenes/manual_usuario/12_historial.png}}
\caption{Historial de reservas}
\end{figure}

\begin{center}\rule{0.5\linewidth}{0.5pt}\end{center}

\subsubsection{6.2 Gestionar
Notificaciones}\label{gestionar-notificaciones}

EstacionaUNSA envía notificaciones para mantenerte informado sobre tus
reservas.

\textbf{Tipos de notificaciones:}

\begin{longtable}[]{@{}
  >{\raggedright\arraybackslash}p{(\linewidth - 4\tabcolsep) * \real{0.1875}}
  >{\raggedright\arraybackslash}p{(\linewidth - 4\tabcolsep) * \real{0.5312}}
  >{\raggedright\arraybackslash}p{(\linewidth - 4\tabcolsep) * \real{0.2812}}@{}}
\toprule\noalign{}
\begin{minipage}[b]{\linewidth}\raggedright
Tipo
\end{minipage} & \begin{minipage}[b]{\linewidth}\raggedright
Cuándo se envía
\end{minipage} & \begin{minipage}[b]{\linewidth}\raggedright
Ejemplo
\end{minipage} \\
\midrule\noalign{}
\endhead
\bottomrule\noalign{}
\endlastfoot
\textbf{Confirmación de reserva} & Al hacer una reserva & ``Reserva
confirmada en Zona A, espacio A-015'' \\
\textbf{Recordatorio} & 5 minutos antes de expirar & ``Tu reserva expira
en 5 minutos'' \\
\textbf{Expiración} & Cuando la reserva expira & ``Tu reserva ha
expirado'' \\
\textbf{Espacio disponible} & Cuando se libera un espacio en tu zona
favorita & ``Espacio disponible en Zona B'' \\
\textbf{Incidencia} & Cuando se reporta una incidencia & ``Se ha
reportado una incidencia en tu cuenta'' \\
\textbf{Sistema} & Mensajes importantes del sistema & ``Mantenimiento
programado el sábado'' \\
\end{longtable}

\textbf{Configurar notificaciones:}

\begin{enumerate}
\def\labelenumi{\arabic{enumi}.}
\item
  \textbf{Ve a tu perfil} y toca \textbf{``Configuración''} o el ícono
  de engranaje (⚙️)
\item
  \textbf{En la sección ``Notificaciones''}, activa o desactiva:

  \begin{itemize}
  \tightlist
  \item
    Notificaciones push
  \item
    Notificaciones de recordatorio
  \item
    Notificaciones de espacios disponibles
  \item
    Notificaciones de sistema
  \end{itemize}
\item
  \textbf{Guarda los cambios}
\end{enumerate}

\begin{quote}
\textbf{ Recomendación:} Mantén activadas al menos las notificaciones
de recordatorio para evitar no-shows.
\end{quote}

\begin{center}\rule{0.5\linewidth}{0.5pt}\end{center}

\subsubsection{6.3 Sistema de
Penalizaciones}\label{sistema-de-penalizaciones}

Para garantizar el uso responsable del sistema, existe un sistema de
penalizaciones por no-shows.

\textbf{¿Qué es un no-show?}

Un ``no-show'' ocurre cuando: - Haces una reserva - No cancelas la
reserva - No llegas al espacio reservado en 15 minutos

\textbf{Niveles de penalización:}

\begin{longtable}[]{@{}lll@{}}
\toprule\noalign{}
No-shows & Consecuencia & Duración \\
\midrule\noalign{}
\endhead
\bottomrule\noalign{}
\endlastfoot
\textbf{1-2} & Sin penalización & - \\
\textbf{3} &  Advertencia oficial & Permanente en tu historial \\
\textbf{5} & 🚫 Suspensión temporal & 7 días sin poder reservar \\
\textbf{10} & 🔒 Suspensión permanente & Cuenta bloqueada \\
\end{longtable}

\textbf{Ver tus penalizaciones:}

\begin{enumerate}
\def\labelenumi{\arabic{enumi}.}
\item
  \textbf{En tu perfil}, revisa la sección \textbf{``Estadísticas''}
\item
  \textbf{Verás:}

  \begin{itemize}
  \tightlist
  \item
    Total de reservas realizadas
  \item
    Reservas completadas
  \item
    Reservas canceladas
  \item
    \textbf{No-shows acumulados}
  \item
    Estado de tu cuenta (Activa, Advertencia, Suspendida)
  \end{itemize}
\end{enumerate}

\textbf{Cómo evitar penalizaciones:}

 \textbf{Haz reservas solo cuando estés cerca} de la universidad\\
 \textbf{Cancela tu reserva} si no puedes llegar\\
 \textbf{Llega a tiempo} dentro de los 15 minutos\\
 \textbf{Verifica tu ubicación} antes de reservar

\begin{quote}
\textbf{ Importante:} Las penalizaciones son automáticas y no se
pueden eliminar. Sé responsable con tus reservas.
\end{quote}

\begin{figure}
\centering
\pandocbounded{\includegraphics[keepaspectratio]{documentacion/imagenes/manual_usuario/13_penalizaciones.png}}
\caption{Penalizaciones}
\end{figure}

\begin{center}\rule{0.5\linewidth}{0.5pt}\end{center}

\subsection{7. Preguntas Frecuentes
(FAQ)}\label{preguntas-frecuentes-faq}

\subsubsection{General}\label{general}

\textbf{P: ¿Quién puede usar EstacionaUNSA?}\\
R: Cualquier miembro de la comunidad UNSA (estudiantes, docentes,
personal administrativo) con un correo institucional @unsa.edu.pe.

\textbf{P: ¿La aplicación es gratuita?}\\
R: Sí, EstacionaUNSA es completamente gratuita para toda la comunidad
UNSA.

\textbf{P: ¿Necesito internet para usar la aplicación?}\\
R: Sí, necesitas conexión a internet (WiFi o datos móviles) para ver la
disponibilidad en tiempo real y hacer reservas.

\subsubsection{Registro y Cuenta}\label{registro-y-cuenta}

\textbf{P: ¿Puedo usar un correo personal?}\\
R: No, solo se permiten correos institucionales con dominio
@unsa.edu.pe.

\textbf{P: ¿Qué hago si no recibo el correo de verificación?}\\
R: Revisa tu carpeta de spam. Si no lo encuentras, intenta registrarte
nuevamente o contacta al soporte.

\textbf{P: ¿Puedo tener múltiples cuentas?}\\
R: No, cada persona debe tener solo una cuenta asociada a su correo
institucional.

\subsubsection{Vehículos}\label{vehuxedculos}

\textbf{P: ¿Cuántos vehículos puedo registrar?}\\
R: Puedes registrar múltiples vehículos en tu cuenta.

\textbf{P: ¿Puedo usar el vehículo de otra persona?}\\
R: Sí, pero debes registrar ese vehículo en tu cuenta antes de hacer una
reserva.

\textbf{P: ¿Qué pasa si cambio de vehículo?}\\
R: Puedes editar o eliminar vehículos en cualquier momento desde tu
perfil.

\subsubsection{Reservas}\label{reservas}

\textbf{P: ¿Puedo reservar con anticipación desde mi casa?}\\
R: No, debes estar a 500 metros o menos de la zona para poder reservar.

\textbf{P: ¿Puedo extender mi reserva de 15 minutos?}\\
R: No, las reservas no son renovables. Tienes 15 minutos para llegar.

\textbf{P: ¿Qué pasa si llego tarde?}\\
R: Si no llegas en 15 minutos, la reserva expirará y se contará como
no-show.

\textbf{P: ¿Puedo reservar para otra persona?}\\
R: No, las reservas son personales y deben coincidir con el vehículo
registrado en tu cuenta.

\textbf{P: ¿Cuánto tiempo puedo estar estacionado?}\\
R: No hay límite de tiempo una vez que ingresas. El límite de 15 minutos
es solo para la reserva.

\subsubsection{Problemas Técnicos}\label{problemas-tuxe9cnicos}

\textbf{P: La aplicación no detecta mi ubicación}\\
R: Asegúrate de tener el GPS activado y haber concedido permisos de
ubicación a la aplicación.

\textbf{P: No puedo hacer una reserva}\\
R: Verifica que: - Estés a 500m o menos de la zona - No tengas otra
reserva activa - Tu cuenta no esté suspendida - Haya espacios
disponibles

\textbf{P: Mi reserva no aparece}\\
R: Cierra y vuelve a abrir la aplicación. Si el problema persiste,
contacta al soporte.

\begin{center}\rule{0.5\linewidth}{0.5pt}\end{center}

\subsection{8. Solución de Problemas}\label{soluciuxf3n-de-problemas}

\subsubsection{Problema: No puedo iniciar
sesión}\label{problema-no-puedo-iniciar-sesiuxf3n}

\textbf{Posibles causas y soluciones:}

\begin{enumerate}
\def\labelenumi{\arabic{enumi}.}
\tightlist
\item
  \textbf{Contraseña incorrecta}

  \begin{itemize}
  \tightlist
  \item
    Verifica que estás escribiendo correctamente
  \item
    Usa la opción ``Olvidé mi contraseña''
  \end{itemize}
\item
  \textbf{Cuenta no verificada}

  \begin{itemize}
  \tightlist
  \item
    Revisa tu correo y verifica tu cuenta
  \item
    Solicita un nuevo correo de verificación
  \end{itemize}
\item
  \textbf{Correo no institucional}

  \begin{itemize}
  \tightlist
  \item
    Asegúrate de usar tu correo @unsa.edu.pe
  \end{itemize}
\end{enumerate}

\subsubsection{Problema: No puedo hacer una
reserva}\label{problema-no-puedo-hacer-una-reserva}

\textbf{Posibles causas y soluciones:}

\begin{enumerate}
\def\labelenumi{\arabic{enumi}.}
\tightlist
\item
  \textbf{Fuera del rango de 500m}

  \begin{itemize}
  \tightlist
  \item
    Acércate más a la universidad
  \item
    Verifica que tu GPS esté activado
  \end{itemize}
\item
  \textbf{Ya tienes una reserva activa}

  \begin{itemize}
  \tightlist
  \item
    Cancela tu reserva actual primero
  \item
    Espera a que expire tu reserva actual
  \end{itemize}
\item
  \textbf{Cuenta suspendida}

  \begin{itemize}
  \tightlist
  \item
    Revisa tus penalizaciones en el perfil
  \item
    Espera a que termine el período de suspensión
  \end{itemize}
\item
  \textbf{No hay espacios disponibles}

  \begin{itemize}
  \tightlist
  \item
    Intenta en otra zona
  \item
    Espera a que se libere un espacio
  \end{itemize}
\end{enumerate}

\subsubsection{Problema: La aplicación se cierra
inesperadamente}\label{problema-la-aplicaciuxf3n-se-cierra-inesperadamente}

\textbf{Soluciones:}

\begin{enumerate}
\def\labelenumi{\arabic{enumi}.}
\tightlist
\item
  \textbf{Actualiza la aplicación} a la última versión
\item
  \textbf{Limpia la caché} de la aplicación en configuración del sistema
\item
  \textbf{Reinicia tu dispositivo}
\item
  \textbf{Reinstala la aplicación} si el problema persiste
\end{enumerate}

\subsubsection{Problema: No recibo
notificaciones}\label{problema-no-recibo-notificaciones}

\textbf{Soluciones:}

\begin{enumerate}
\def\labelenumi{\arabic{enumi}.}
\tightlist
\item
  \textbf{Verifica los permisos:}

  \begin{itemize}
  \tightlist
  \item
    Ve a Configuración del sistema
  \item
    Busca EstacionaUNSA
  \item
    Asegúrate de que las notificaciones estén activadas
  \end{itemize}
\item
  \textbf{Revisa la configuración de la app:}

  \begin{itemize}
  \tightlist
  \item
    Abre EstacionaUNSA
  \item
    Ve a Configuración
  \item
    Activa las notificaciones
  \end{itemize}
\item
  \textbf{Verifica tu conexión a internet}
\end{enumerate}

\subsubsection{Problema: El mapa no
carga}\label{problema-el-mapa-no-carga}

\textbf{Soluciones:}

\begin{enumerate}
\def\labelenumi{\arabic{enumi}.}
\tightlist
\item
  \textbf{Verifica tu conexión a internet}
\item
  \textbf{Activa los servicios de ubicación}
\item
  \textbf{Concede permisos de ubicación} a la aplicación
\item
  \textbf{Reinicia la aplicación}
\end{enumerate}

\begin{center}\rule{0.5\linewidth}{0.5pt}\end{center}

\subsection{9. Contacto y Soporte}\label{contacto-y-soporte}

\subsubsection{Canales de Soporte}\label{canales-de-soporte}

Si tienes problemas técnicos o preguntas que no están cubiertas en este
manual:

\textbf{ Correo Electrónico:}\\
soporte.estacionaunsa@unsa.edu.pe

\begin{center}\rule{0.5\linewidth}{0.5pt}\end{center}

\subsection{Información del Proyecto}\label{informaciuxf3n-del-proyecto}

\textbf{Repositorio GitHub:}\\
https://github.com/Choflis/EstacionaUNSA.git

\textbf{Descarga de la aplicación:}\\
El enlace de descarga del APK y el video demostrativo estarán
disponibles en Google Drive.

\textbf{Equipo de Desarrollo:} - Luis Guillermo Luque Condori - Líder de
Proyecto / Desarrollador Flutter - Dennis Javier Quispe Saavedra -
Diseño UI/UX - Fernando Miguel Garambel Marín - Backend \& Firebase

\begin{center}\rule{0.5\linewidth}{0.5pt}\end{center}

\subsection{Licencia y Términos de
Uso}\label{licencia-y-tuxe9rminos-de-uso}

EstacionaUNSA es un proyecto académico desarrollado en la Universidad
Nacional de San Agustín de Arequipa con fines educativos. El uso de esta
aplicación está sujeto a las políticas y reglamentos de la UNSA.

\end{document}
