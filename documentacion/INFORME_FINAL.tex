% Options for packages loaded elsewhere
\PassOptionsToPackage{unicode}{hyperref}
\PassOptionsToPackage{hyphens}{url}
\documentclass[
]{article}
\usepackage{xcolor}
\usepackage{amsmath,amssymb}
\setcounter{secnumdepth}{5}
\usepackage{iftex}
\ifPDFTeX
  \usepackage[T1]{fontenc}
  \usepackage[utf8]{inputenc}
  \usepackage{textcomp} % provide euro and other symbols
\else % if luatex or xetex
  \usepackage{unicode-math} % this also loads fontspec
  \defaultfontfeatures{Scale=MatchLowercase}
  \defaultfontfeatures[\rmfamily]{Ligatures=TeX,Scale=1}
\fi
\usepackage{lmodern}
\ifPDFTeX\else
  % xetex/luatex font selection
\fi
% Use upquote if available, for straight quotes in verbatim environments
\IfFileExists{upquote.sty}{\usepackage{upquote}}{}
\IfFileExists{microtype.sty}{% use microtype if available
  \usepackage[]{microtype}
  \UseMicrotypeSet[protrusion]{basicmath} % disable protrusion for tt fonts
}{}
\makeatletter
\@ifundefined{KOMAClassName}{% if non-KOMA class
  \IfFileExists{parskip.sty}{%
    \usepackage{parskip}
  }{% else
    \setlength{\parindent}{0pt}
    \setlength{\parskip}{6pt plus 2pt minus 1pt}}
}{% if KOMA class
  \KOMAoptions{parskip=half}}
\makeatother
\usepackage{color}
\usepackage{fancyvrb}
\newcommand{\VerbBar}{|}
\newcommand{\VERB}{\Verb[commandchars=\\\{\}]}
\DefineVerbatimEnvironment{Highlighting}{Verbatim}{commandchars=\\\{\}}
% Add ',fontsize=\small' for more characters per line
\newenvironment{Shaded}{}{}
\newcommand{\AlertTok}[1]{\textcolor[rgb]{1.00,0.00,0.00}{\textbf{#1}}}
\newcommand{\AnnotationTok}[1]{\textcolor[rgb]{0.38,0.63,0.69}{\textbf{\textit{#1}}}}
\newcommand{\AttributeTok}[1]{\textcolor[rgb]{0.49,0.56,0.16}{#1}}
\newcommand{\BaseNTok}[1]{\textcolor[rgb]{0.25,0.63,0.44}{#1}}
\newcommand{\BuiltInTok}[1]{\textcolor[rgb]{0.00,0.50,0.00}{#1}}
\newcommand{\CharTok}[1]{\textcolor[rgb]{0.25,0.44,0.63}{#1}}
\newcommand{\CommentTok}[1]{\textcolor[rgb]{0.38,0.63,0.69}{\textit{#1}}}
\newcommand{\CommentVarTok}[1]{\textcolor[rgb]{0.38,0.63,0.69}{\textbf{\textit{#1}}}}
\newcommand{\ConstantTok}[1]{\textcolor[rgb]{0.53,0.00,0.00}{#1}}
\newcommand{\ControlFlowTok}[1]{\textcolor[rgb]{0.00,0.44,0.13}{\textbf{#1}}}
\newcommand{\DataTypeTok}[1]{\textcolor[rgb]{0.56,0.13,0.00}{#1}}
\newcommand{\DecValTok}[1]{\textcolor[rgb]{0.25,0.63,0.44}{#1}}
\newcommand{\DocumentationTok}[1]{\textcolor[rgb]{0.73,0.13,0.13}{\textit{#1}}}
\newcommand{\ErrorTok}[1]{\textcolor[rgb]{1.00,0.00,0.00}{\textbf{#1}}}
\newcommand{\ExtensionTok}[1]{#1}
\newcommand{\FloatTok}[1]{\textcolor[rgb]{0.25,0.63,0.44}{#1}}
\newcommand{\FunctionTok}[1]{\textcolor[rgb]{0.02,0.16,0.49}{#1}}
\newcommand{\ImportTok}[1]{\textcolor[rgb]{0.00,0.50,0.00}{\textbf{#1}}}
\newcommand{\InformationTok}[1]{\textcolor[rgb]{0.38,0.63,0.69}{\textbf{\textit{#1}}}}
\newcommand{\KeywordTok}[1]{\textcolor[rgb]{0.00,0.44,0.13}{\textbf{#1}}}
\newcommand{\NormalTok}[1]{#1}
\newcommand{\OperatorTok}[1]{\textcolor[rgb]{0.40,0.40,0.40}{#1}}
\newcommand{\OtherTok}[1]{\textcolor[rgb]{0.00,0.44,0.13}{#1}}
\newcommand{\PreprocessorTok}[1]{\textcolor[rgb]{0.74,0.48,0.00}{#1}}
\newcommand{\RegionMarkerTok}[1]{#1}
\newcommand{\SpecialCharTok}[1]{\textcolor[rgb]{0.25,0.44,0.63}{#1}}
\newcommand{\SpecialStringTok}[1]{\textcolor[rgb]{0.73,0.40,0.53}{#1}}
\newcommand{\StringTok}[1]{\textcolor[rgb]{0.25,0.44,0.63}{#1}}
\newcommand{\VariableTok}[1]{\textcolor[rgb]{0.10,0.09,0.49}{#1}}
\newcommand{\VerbatimStringTok}[1]{\textcolor[rgb]{0.25,0.44,0.63}{#1}}
\newcommand{\WarningTok}[1]{\textcolor[rgb]{0.38,0.63,0.69}{\textbf{\textit{#1}}}}
\usepackage{longtable,booktabs,array}
\usepackage{calc} % for calculating minipage widths
% Correct order of tables after \paragraph or \subparagraph
\usepackage{etoolbox}
\makeatletter
\patchcmd\longtable{\par}{\if@noskipsec\mbox{}\fi\par}{}{}
\makeatother
% Allow footnotes in longtable head/foot
\IfFileExists{footnotehyper.sty}{\usepackage{footnotehyper}}{\usepackage{footnote}}
\makesavenoteenv{longtable}
\ifLuaTeX
\usepackage[bidi=basic]{babel}
\else
\usepackage[bidi=default]{babel}
\fi
\babelprovide[main,import]{spanish}
% get rid of language-specific shorthands (see #6817):
\let\LanguageShortHands\languageshorthands
\def\languageshorthands#1{}
\setlength{\emergencystretch}{3em} % prevent overfull lines
\providecommand{\tightlist}{%
  \setlength{\itemsep}{0pt}\setlength{\parskip}{0pt}}
\usepackage{bookmark}
\IfFileExists{xurl.sty}{\usepackage{xurl}}{} % add URL line breaks if available
\urlstyle{same}
\hypersetup{
  pdflang={es},
  hidelinks,
  pdfcreator={LaTeX via pandoc}}

\author{}
\date{}

\begin{document}

{
\setcounter{tocdepth}{3}
\tableofcontents
}
\section{ Informe Final -
EstacionaUNSA}\label{informe-final---estacionaunsa}

\textbf{Sistema de Gestión de Estacionamientos Universitarios}

\begin{center}\rule{0.5\linewidth}{0.5pt}\end{center}

\textbf{Universidad Nacional de San Agustín de Arequipa}\\
\textbf{Escuela Profesional de Ingeniería de Sistemas}\\
\textbf{Curso:} Construcción de Software

\textbf{Equipo de Desarrollo:} - Luis Guillermo Luque Condori - Líder de
Proyecto / Desarrollador Flutter - Dennis Javier Quispe Saavedra -
Diseño UI/UX\\
- Fernando Miguel Garambel Marín - Backend \& Firebase

\textbf{Versión:} 1.0\\
\textbf{Fecha:} Diciembre 2025

\begin{center}\rule{0.5\linewidth}{0.5pt}\end{center}

\subsection{ Índice}\label{uxedndice}

\begin{enumerate}
\def\labelenumi{\arabic{enumi}.}
\tightlist
\item
  \hyperref[1-resumen-ejecutivo]{Resumen Ejecutivo}
\item
  \hyperref[2-enlaces-del-proyecto]{Enlaces del Proyecto}
\item
  \hyperref[3-introducciuxf3n]{Introducción}
\item
  \hyperref[4-proceso-de-desarrollo]{Proceso de Desarrollo}
\item
  \hyperref[5-tecnologuxedas-empleadas]{Tecnologías Empleadas}
\item
  \hyperref[6-arquitectura-del-sistema]{Arquitectura del Sistema}
\item
  \hyperref[7-funcionalidades-implementadas]{Funcionalidades
  Implementadas}
\item
  \hyperref[8-interfaces-de-usuario]{Interfaces de Usuario}
\item
  \hyperref[9-base-de-datos]{Base de Datos}
\item
  \hyperref[10-testing-y-calidad]{Testing y Calidad}
\item
  \hyperref[11-mantenimiento]{Mantenimiento}
\item
  \hyperref[12-lecciones-aprendidas]{Lecciones Aprendidas}
\item
  \hyperref[13-conclusiones]{Conclusiones}
\item
  \hyperref[14-referencias]{Referencias}
\end{enumerate}

\begin{center}\rule{0.5\linewidth}{0.5pt}\end{center}

\subsection{1. Resumen Ejecutivo}\label{resumen-ejecutivo}

EstacionaUNSA es una solución móvil multiplataforma desarrollada con
Flutter para optimizar la gestión de estacionamientos en la Universidad
Nacional de San Agustín de Arequipa. El proyecto aborda la problemática
de la falta de información en tiempo real sobre la disponibilidad de
espacios de estacionamiento, lo que genera pérdida de tiempo y
congestión vehicular en el campus universitario.

\subsubsection{Objetivos Alcanzados}\label{objetivos-alcanzados}

 \textbf{Visualización en tiempo real} de la disponibilidad de
espacios en 3 zonas principales\\
 \textbf{Sistema de reservas} con restricciones de distancia y
tiempo\\
 \textbf{Autenticación institucional} exclusiva para la comunidad
UNSA\\
 \textbf{Gestión de vehículos} con registro de múltiples unidades\\
 \textbf{Historial completo} de entradas y salidas\\
 \textbf{Sistema de notificaciones} push para alertas importantes\\
 \textbf{Control de penalizaciones} automático para uso responsable

\subsubsection{Resultados Clave}\label{resultados-clave}

\begin{itemize}
\tightlist
\item
  \textbf{120 espacios} de estacionamiento gestionados en 3 zonas
\item
  \textbf{Arquitectura escalable} basada en Clean Architecture y
  Provider Pattern
\item
  \textbf{Base de datos en tiempo real} con Firebase Firestore
\item
  \textbf{Plan de testing completo} con 5 fases documentadas
\item
  \textbf{30 errores críticos} corregidos durante el análisis estático
\end{itemize}

\begin{center}\rule{0.5\linewidth}{0.5pt}\end{center}

\subsection{2. Enlaces del Proyecto}\label{enlaces-del-proyecto}

\subsubsection{Repositorio de Código}\label{repositorio-de-cuxf3digo}

\textbf{GitHub:}\\
https://github.com/Choflis/EstacionaUNSA.git

El repositorio contiene: - Código fuente completo de la aplicación
Flutter - Documentación técnica y de desarrollo - Plan de testing y
evidencias - Configuración de Firebase - Reglas de seguridad de
Firestore

\subsubsection{Aplicación Compilada y
Video}\label{aplicaciuxf3n-compilada-y-video}

\textbf{Google Drive:}\\
{[}Enlace pendiente - Se proporcionará el enlace al Drive con el APK y
video demostrativo{]}

El Drive incluirá: - \texttt{estaciona-unsa.apk} - Aplicación compilada
para Android - \texttt{demo-estacionaunsa.mp4} - Video demostrativo de
las funcionalidades

\begin{center}\rule{0.5\linewidth}{0.5pt}\end{center}

\subsection{3. Introducción}\label{introducciuxf3n}

\subsubsection{3.1 Contexto y
Problemática}\label{contexto-y-problemuxe1tica}

La Universidad Nacional de San Agustín de Arequipa cuenta con múltiples
zonas de estacionamiento distribuidas en su campus. Sin embargo, los
usuarios (estudiantes, docentes y personal administrativo) enfrentan los
siguientes problemas:

\begin{enumerate}
\def\labelenumi{\arabic{enumi}.}
\tightlist
\item
  \textbf{Falta de información en tiempo real} sobre espacios
  disponibles
\item
  \textbf{Pérdida de tiempo} buscando estacionamiento
\item
  \textbf{Congestión vehicular} en horas pico
\item
  \textbf{Ausencia de control} sobre el uso de espacios
\item
  \textbf{Dificultad para planificar} la llegada al campus
\end{enumerate}

\subsubsection{3.2 Objetivos del Proyecto}\label{objetivos-del-proyecto}

\paragraph{Objetivo General}\label{objetivo-general}

Desarrollar una aplicación móvil multiplataforma que permita a la
comunidad UNSA visualizar, reservar y gestionar espacios de
estacionamiento en tiempo real, mejorando la movilidad interna y el
control del flujo vehicular.

\paragraph{Objetivos Específicos}\label{objetivos-especuxedficos}

\begin{enumerate}
\def\labelenumi{\arabic{enumi}.}
\tightlist
\item
  Implementar un sistema de autenticación seguro con correos
  institucionales
\item
  Desarrollar un módulo de visualización en tiempo real de espacios
  disponibles
\item
  Crear un sistema de reservas con restricciones de distancia y tiempo
\item
  Diseñar una interfaz intuitiva y moderna basada en Material Design 3
\item
  Implementar un sistema de notificaciones push
\item
  Desarrollar un módulo de historial y estadísticas de uso
\item
  Establecer un sistema de penalizaciones para uso responsable
\end{enumerate}

\subsubsection{3.3 Alcance}\label{alcance}

\textbf{Incluye:} - Aplicación móvil para Android (iOS en desarrollo
futuro) - Gestión de 3 zonas principales de estacionamiento (120
espacios totales) - Sistema de reservas con duración de 15 minutos -
Autenticación con Firebase Authentication - Base de datos en tiempo real
con Firestore - Notificaciones push con Firebase Cloud Messaging -
Historial completo de uso

\textbf{No incluye:} - Integración con cámaras de seguridad - Sensores
IoT para detección automática de vehículos - Sistema de pagos -
Aplicación web administrativa (en roadmap futuro) - Integración con
sistemas académicos de la UNSA

\begin{center}\rule{0.5\linewidth}{0.5pt}\end{center}

\subsection{4. Proceso de Desarrollo}\label{proceso-de-desarrollo}

\subsubsection{4.1 Metodología
Utilizada}\label{metodologuxeda-utilizada}

El proyecto se desarrolló utilizando una \textbf{metodología ágil
adaptada} con sprints de 1-2 semanas, permitiendo iteraciones rápidas y
ajustes basados en retroalimentación.

\paragraph{Principios Aplicados}\label{principios-aplicados}

\begin{itemize}
\tightlist
\item
  \textbf{Desarrollo iterativo e incremental}
\item
  \textbf{Comunicación constante} entre miembros del equipo
\item
  \textbf{Documentación continua} de decisiones técnicas
\item
  \textbf{Testing desde etapas tempranas}
\item
  \textbf{Revisión de código} mediante pull requests
\end{itemize}

\subsubsection{4.2 Fases del Proyecto}\label{fases-del-proyecto}

\begin{Shaded}
\begin{Highlighting}[]
\NormalTok{gantt}
\NormalTok{    title Cronograma de Desarrollo EstacionaUNSA}
\NormalTok{    dateFormat YYYY{-}MM{-}DD}
\NormalTok{    section Planificación}
\NormalTok{    Análisis de requisitos       :2025{-}09{-}01, 7d}
\NormalTok{    Diseño de arquitectura       :2025{-}09{-}08, 7d}
\NormalTok{    Diseño de base de datos      :2025{-}09{-}15, 5d}
\NormalTok{    section Desarrollo}
\NormalTok{    Configuración Firebase        :2025{-}09{-}20, 3d}
\NormalTok{    Autenticación                :2025{-}09{-}23, 7d}
\NormalTok{    Gestión de vehículos         :2025{-}09{-}30, 7d}
\NormalTok{    Sistema de zonas             :2025{-}10{-}07, 10d}
\NormalTok{    Sistema de reservas          :2025{-}10{-}17, 14d}
\NormalTok{    Notificaciones               :2025{-}10{-}31, 7d}
\NormalTok{    Historial                    :2025{-}11{-}07, 7d}
\NormalTok{    section Testing}
\NormalTok{    Análisis estático            :2025{-}11{-}14, 3d}
\NormalTok{    Plan de pruebas              :2025{-}11{-}17, 5d}
\NormalTok{    Casos de prueba              :2025{-}11{-}22, 5d}
\NormalTok{    Ejecución de pruebas         :2025{-}11{-}27, 3d}
\NormalTok{    section Documentación}
\NormalTok{    Manual de usuario            :2025{-}12{-}01, 5d}
\NormalTok{    Informe final                :2025{-}12{-}06, 5d}
\end{Highlighting}
\end{Shaded}

\subsubsection{4.3 Distribución de
Tareas}\label{distribuciuxf3n-de-tareas}

\begin{longtable}[]{@{}
  >{\raggedright\arraybackslash}p{(\linewidth - 2\tabcolsep) * \real{0.2308}}
  >{\raggedright\arraybackslash}p{(\linewidth - 2\tabcolsep) * \real{0.7692}}@{}}
\toprule\noalign{}
\begin{minipage}[b]{\linewidth}\raggedright
Miembro
\end{minipage} & \begin{minipage}[b]{\linewidth}\raggedright
Responsabilidades Principales
\end{minipage} \\
\midrule\noalign{}
\endhead
\bottomrule\noalign{}
\endlastfoot
\textbf{Luis Guillermo Luque Condori} & • Líder de proyecto y
coordinación• Desarrollo de UI/UX con Flutter• Implementación de
providers• Sistema de navegación• Plan de pruebas y QA \\
\textbf{Dennis Javier Quispe Saavedra} & • Diseño de interfaces y
experiencia de usuario• Implementación de widgets personalizados•
Testing de componentes UI• Diseño de flujos de usuario• Pruebas
funcionales \\
\textbf{Fernando Miguel Garambel Marín} & • Configuración y gestión de
Firebase• Desarrollo de servicios backend• Implementación de Firestore•
Reglas de seguridad• Testing de integración y backend \\
\end{longtable}

\begin{center}\rule{0.5\linewidth}{0.5pt}\end{center}

\subsection{5. Tecnologías Empleadas}\label{tecnologuxedas-empleadas}

\subsubsection{5.1 Stack Tecnológico
Completo}\label{stack-tecnoluxf3gico-completo}

\paragraph{Frontend - Aplicación
Móvil}\label{frontend---aplicaciuxf3n-muxf3vil}

\begin{longtable}[]{@{}
  >{\raggedright\arraybackslash}p{(\linewidth - 4\tabcolsep) * \real{0.3750}}
  >{\raggedright\arraybackslash}p{(\linewidth - 4\tabcolsep) * \real{0.2812}}
  >{\raggedright\arraybackslash}p{(\linewidth - 4\tabcolsep) * \real{0.3438}}@{}}
\toprule\noalign{}
\begin{minipage}[b]{\linewidth}\raggedright
Tecnología
\end{minipage} & \begin{minipage}[b]{\linewidth}\raggedright
Versión
\end{minipage} & \begin{minipage}[b]{\linewidth}\raggedright
Propósito
\end{minipage} \\
\midrule\noalign{}
\endhead
\bottomrule\noalign{}
\endlastfoot
\textbf{Flutter} & 3.24.5 & Framework principal para desarrollo
multiplataforma \\
\textbf{Dart} & 3.5.4 & Lenguaje de programación \\
\textbf{Material Design 3} & Latest & Sistema de diseño para UI/UX \\
\textbf{Provider} & 6.1.2 & Gestión de estado \\
\textbf{Google Maps Flutter} & 2.9.0 & Visualización de mapas \\
\textbf{Geolocator} & 13.0.2 & Servicios de geolocalización \\
\end{longtable}

\paragraph{Backend - Firebase}\label{backend---firebase}

\begin{longtable}[]{@{}
  >{\raggedright\arraybackslash}p{(\linewidth - 2\tabcolsep) * \real{0.4762}}
  >{\raggedright\arraybackslash}p{(\linewidth - 2\tabcolsep) * \real{0.5238}}@{}}
\toprule\noalign{}
\begin{minipage}[b]{\linewidth}\raggedright
Servicio
\end{minipage} & \begin{minipage}[b]{\linewidth}\raggedright
Propósito
\end{minipage} \\
\midrule\noalign{}
\endhead
\bottomrule\noalign{}
\endlastfoot
\textbf{Firebase Authentication} & Autenticación de usuarios con
email/password \\
\textbf{Cloud Firestore} & Base de datos NoSQL en tiempo real \\
\textbf{Firebase Cloud Messaging} & Notificaciones push \\
\textbf{Cloud Storage} & Almacenamiento de imágenes de vehículos \\
\textbf{Firebase Analytics} & Análisis de uso (opcional) \\
\end{longtable}

\paragraph{Herramientas de Desarrollo}\label{herramientas-de-desarrollo}

\begin{longtable}[]{@{}ll@{}}
\toprule\noalign{}
Herramienta & Propósito \\
\midrule\noalign{}
\endhead
\bottomrule\noalign{}
\endlastfoot
\textbf{Visual Studio Code} & IDE principal \\
\textbf{Android Studio} & Emuladores y debugging Android \\
\textbf{Git / GitHub} & Control de versiones \\
\textbf{Flutter DevTools} & Debugging y profiling \\
\textbf{Postman} & Testing de APIs (si aplica) \\
\textbf{Firebase Console} & Gestión de servicios Firebase \\
\end{longtable}

\paragraph{Testing y Calidad}\label{testing-y-calidad}

\begin{longtable}[]{@{}ll@{}}
\toprule\noalign{}
Herramienta & Propósito \\
\midrule\noalign{}
\endhead
\bottomrule\noalign{}
\endlastfoot
\textbf{Dart Analyzer} & Análisis estático de código \\
\textbf{Flutter Test} & Testing unitario y de widgets \\
\textbf{Integration Test} & Testing de integración \\
\textbf{Firebase Test Lab} & Testing en dispositivos reales (futuro) \\
\end{longtable}

\subsubsection{5.2 Justificación de Elecciones
Tecnológicas}\label{justificaciuxf3n-de-elecciones-tecnoluxf3gicas}

\paragraph{¿Por qué Flutter?}\label{por-quuxe9-flutter}

 \textbf{Multiplataforma:} Un solo código para Android e iOS\\
 \textbf{Rendimiento nativo:} Compilación directa a código nativo\\
 \textbf{Hot Reload:} Desarrollo rápido con recarga en caliente\\
 \textbf{Material Design 3:} Componentes modernos integrados\\
 \textbf{Comunidad activa:} Gran ecosistema de paquetes\\
 \textbf{Documentación excelente:} Recursos de aprendizaje abundantes

\paragraph{¿Por qué Firebase?}\label{por-quuxe9-firebase}

 \textbf{Tiempo real:} Sincronización instantánea de datos\\
 \textbf{Escalabilidad:} Crece con las necesidades del proyecto\\
 \textbf{Sin servidor:} No requiere infraestructura propia\\
 \textbf{Seguridad:} Reglas de seguridad robustas\\
 \textbf{Integración:} SDK oficial para Flutter\\
 \textbf{Gratuito:} Plan generoso para proyectos académicos

\paragraph{¿Por qué Provider para gestión de
estado?}\label{por-quuxe9-provider-para-gestiuxf3n-de-estado}

 \textbf{Simplicidad:} Curva de aprendizaje suave\\
 \textbf{Recomendado:} Solución oficial de Flutter\\
 \textbf{Eficiencia:} Reconstrucción selectiva de widgets\\
 \textbf{Escalable:} Adecuado para proyectos medianos\\
 \textbf{Testeable:} Facilita el testing unitario

\begin{center}\rule{0.5\linewidth}{0.5pt}\end{center}

\subsection{6. Arquitectura del Sistema}\label{arquitectura-del-sistema}

\subsubsection{6.1 Arquitectura General}\label{arquitectura-general}

EstacionaUNSA implementa \textbf{Clean Architecture} combinada con el
\textbf{Provider Pattern} para separación de responsabilidades y
mantenibilidad.

\begin{Shaded}
\begin{Highlighting}[]
\NormalTok{flowchart TB}
\NormalTok{    subgraph "Presentation Layer"}
\NormalTok{        A[Screens] {-}{-}\textgreater{} B[Widgets]}
\NormalTok{    end}
    
\NormalTok{    subgraph "State Management"}
\NormalTok{        C[Providers]}
\NormalTok{    end}
    
\NormalTok{    subgraph "Business Logic"}
\NormalTok{        D[Services]}
\NormalTok{        E[Models]}
\NormalTok{    end}
    
\NormalTok{    subgraph "Data Layer"}
\NormalTok{        F[Firebase Auth]}
\NormalTok{        G[Cloud Firestore]}
\NormalTok{        H[Cloud Messaging]}
\NormalTok{        I[Cloud Storage]}
\NormalTok{    end}
    
\NormalTok{    A {-}{-}\textgreater{} C}
\NormalTok{    B {-}{-}\textgreater{} C}
\NormalTok{    C {-}{-}\textgreater{} D}
\NormalTok{    D {-}{-}\textgreater{} E}
\NormalTok{    D {-}{-}\textgreater{} F}
\NormalTok{    D {-}{-}\textgreater{} G}
\NormalTok{    D {-}{-}\textgreater{} H}
\NormalTok{    D {-}{-}\textgreater{} I}
\end{Highlighting}
\end{Shaded}

\subsubsection{6.2 Estructura del
Proyecto}\label{estructura-del-proyecto}

\begin{verbatim}
estaciona_unsa/
├── lib/
│   ├── main.dart                 # Punto de entrada
│   ├── firebase_options.dart     # Configuración Firebase
│   ├── config/                   # Configuraciones
│   │   ├── theme.dart
│   │   └── routes.dart
│   ├── models/                   # Modelos de datos
│   │   ├── user_model.dart
│   │   ├── vehicle_model.dart
│   │   ├── parking_zone_model.dart
│   │   ├── parking_spot_model.dart
│   │   └── reservation_model.dart
│   ├── providers/                # Gestión de estado
│   │   ├── auth_provider.dart
│   │   ├── parking_provider.dart
│   │   ├── reservation_provider.dart
│   │   ├── vehicle_provider.dart
│   │   └── notification_provider.dart
│   ├── services/                 # Lógica de negocio
│   │   ├── firebase/
│   │   │   ├── auth_service.dart
│   │   │   ├── firestore_service.dart
│   │   │   ├── messaging_service.dart
│   │   │   └── storage_service.dart
│   │   └── location_service.dart
│   ├── screens/                  # Pantallas
│   │   ├── auth_wrapper.dart
│   │   ├── login_screen.dart
│   │   ├── home_screen.dart
│   │   ├── map_screen.dart
│   │   ├── my_reservation_screen.dart
│   │   ├── my_vehicle_screen.dart
│   │   ├── history_screen.dart
│   │   └── profile/
│   │       └── profile_screen.dart
│   ├── widgets/                  # Componentes reutilizables
│   │   ├── zone_card.dart
│   │   ├── spot_grid.dart
│   │   ├── reservation_card.dart
│   │   └── vehicle_card.dart
│   └── utils/                    # Utilidades
│       ├── constants.dart
│       ├── validators.dart
│       └── helpers.dart
├── assets/                       # Recursos
├── test/                         # Tests
└── pubspec.yaml                  # Dependencias
\end{verbatim}

\subsubsection{6.3 Patrones de Diseño
Utilizados}\label{patrones-de-diseuxf1o-utilizados}

\paragraph{1. Provider Pattern (State
Management)}\label{provider-pattern-state-management}

Gestión centralizada del estado de la aplicación con notificación
automática a widgets suscritos.

\textbf{Ejemplo:}

\begin{Shaded}
\begin{Highlighting}[]
\KeywordTok{class}\NormalTok{ ReservationProvider }\KeywordTok{extends}\NormalTok{ ChangeNotifier }\OperatorTok{\{}
\NormalTok{  ReservationModel}\OperatorTok{?}\NormalTok{ \_activeReservation;}
  
  \DataTypeTok{Future}\OperatorTok{\textless{}}\DataTypeTok{void}\OperatorTok{\textgreater{}}\NormalTok{ createReservation(}\DataTypeTok{String}\NormalTok{ spotId) }\AttributeTok{async} \OperatorTok{\{}
    \CommentTok{// Lógica de creación}
\NormalTok{    notifyListeners(); }\CommentTok{// Notifica a los widgets}
  \OperatorTok{\}}
\OperatorTok{\}}
\end{Highlighting}
\end{Shaded}

\paragraph{2. Repository Pattern (Data
Access)}\label{repository-pattern-data-access}

Abstracción del acceso a datos para facilitar testing y cambios futuros.

\textbf{Ejemplo:}

\begin{Shaded}
\begin{Highlighting}[]
\KeywordTok{class}\NormalTok{ FirestoreService }\OperatorTok{\{}
  \DataTypeTok{Future}\OperatorTok{\textless{}}\DataTypeTok{List}\OperatorTok{\textless{}}\NormalTok{ParkingZone}\OperatorTok{\textgreater{}\textgreater{}}\NormalTok{ getAllZones() }\AttributeTok{async} \OperatorTok{\{}
    \CommentTok{// Acceso a Firestore}
  \OperatorTok{\}}
\OperatorTok{\}}
\end{Highlighting}
\end{Shaded}

\paragraph{3. Singleton Pattern}\label{singleton-pattern}

Instancias únicas de servicios críticos.

\textbf{Ejemplo:}

\begin{Shaded}
\begin{Highlighting}[]
\KeywordTok{class}\NormalTok{ LocationService }\OperatorTok{\{}
  \AttributeTok{static} \AttributeTok{final}\NormalTok{ LocationService \_instance }\OperatorTok{=}\NormalTok{ LocationService}\OperatorTok{.}\NormalTok{\_internal();}
  \KeywordTok{factory}\NormalTok{ LocationService() }\OperatorTok{=\textgreater{}}\NormalTok{ \_instance;}
\NormalTok{  LocationService}\OperatorTok{.}\NormalTok{\_internal();}
\OperatorTok{\}}
\end{Highlighting}
\end{Shaded}

\begin{center}\rule{0.5\linewidth}{0.5pt}\end{center}

\subsection{7. Funcionalidades
Implementadas}\label{funcionalidades-implementadas}

\subsubsection{7.1 Módulo de
Autenticación}\label{muxf3dulo-de-autenticaciuxf3n}

\textbf{Funcionalidades:} -  Registro con correo institucional
(@unsa.edu.pe) -  Inicio de sesión con email/password - 
Recuperación de contraseña -  Verificación de correo electrónico - 
Cierre de sesión -  Persistencia de sesión

\textbf{Validaciones:} - Solo correos con dominio @unsa.edu.pe -
Contraseña mínima de 6 caracteres - Verificación de correo obligatoria

\subsubsection{7.2 Módulo de Gestión de
Vehículos}\label{muxf3dulo-de-gestiuxf3n-de-vehuxedculos}

\textbf{Funcionalidades:} -  Agregar vehículos (auto/motocicleta) - 
Editar información de vehículos -  Eliminar vehículos -  Subir foto
del vehículo -  Validación de placas

\textbf{Campos:} - Placa (obligatorio) - Tipo (auto/motocicleta) -
Modelo - Color - Foto

\subsubsection{7.3 Módulo de Zonas de
Estacionamiento}\label{muxf3dulo-de-zonas-de-estacionamiento}

\textbf{Funcionalidades:} -  Visualización de 3 zonas principales - 
Información en tiempo real de disponibilidad -  Cálculo de distancia
desde ubicación actual -  Indicadores visuales de capacidad - 
Horarios de operación

\textbf{Zonas implementadas:} 1. Zona A - Entrada Principal (50
espacios) 2. Zona B - Biblioteca Central (30 espacios) 3. Zona C -
Ingenierías (40 espacios)

\subsubsection{7.4 Módulo de Reservas}\label{muxf3dulo-de-reservas}

\textbf{Funcionalidades:} -  Crear reserva (con restricciones) - 
Cancelar reserva -  Ver reserva activa -  Temporizador de expiración
-  Validación de distancia (≤500m) -  Validación de disponibilidad

\textbf{Restricciones:} - Máximo 1 reserva activa por usuario - Duración
de 15 minutos - Distancia máxima de 500 metros - Solo en horarios de
operación

\subsubsection{7.5 Módulo de Historial}\label{muxf3dulo-de-historial}

\textbf{Funcionalidades:} -  Ver historial completo de reservas - 
Filtrar por fecha -  Filtrar por estado -  Ver detalles de cada
reserva -  Estadísticas de uso

\textbf{Estados de reserva:} - Completada - Cancelada - Expirada
(no-show)

\subsubsection{7.6 Módulo de
Notificaciones}\label{muxf3dulo-de-notificaciones}

\textbf{Funcionalidades:} -  Notificaciones push -  Confirmación de
reserva -  Recordatorio (5 min antes de expirar) -  Notificación de
expiración -  Alertas de sistema

\subsubsection{7.7 Sistema de
Penalizaciones}\label{sistema-de-penalizaciones}

\textbf{Funcionalidades:} -  Contador automático de no-shows - 
Advertencias progresivas -  Suspensión temporal (7 días) - 
Suspensión permanente -  Visualización de estadísticas

\textbf{Niveles:} - 3 no-shows: Advertencia - 5 no-shows: Suspensión 7
días - 10 no-shows: Suspensión permanente

\begin{center}\rule{0.5\linewidth}{0.5pt}\end{center}

\subsection{8. Interfaces de Usuario}\label{interfaces-de-usuario}

\subsubsection{8.1 Diseño UI/UX}\label{diseuxf1o-uiux}

El diseño de EstacionaUNSA sigue los principios de \textbf{Material
Design 3}, priorizando:

\begin{itemize}
\tightlist
\item
  \textbf{Simplicidad:} Interfaces limpias y fáciles de entender
\item
  \textbf{Consistencia:} Patrones visuales coherentes
\item
  \textbf{Accesibilidad:} Contraste adecuado y tamaños de fuente
  legibles
\item
  \textbf{Feedback visual:} Indicadores claros de estado y acciones
\end{itemize}

\subsubsection{8.2 Pantallas Principales}\label{pantallas-principales}

\paragraph{1. Pantalla de
Autenticación}\label{pantalla-de-autenticaciuxf3n}

\begin{itemize}
\tightlist
\item
  Login con email/password
\item
  Enlace a registro
\item
  Recuperación de contraseña
\item
  Validación en tiempo real
\end{itemize}

\paragraph{2. Pantalla Principal (Home)}\label{pantalla-principal-home}

\begin{itemize}
\tightlist
\item
  Tarjetas de zonas de estacionamiento
\item
  Indicadores de disponibilidad
\item
  Distancia a cada zona
\item
  Acceso rápido a reservas
\end{itemize}

\paragraph{3. Pantalla de Mapa}\label{pantalla-de-mapa}

\begin{itemize}
\tightlist
\item
  Mapa interactivo con Google Maps
\item
  Marcadores de zonas
\item
  Ubicación del usuario
\item
  Círculo de 500m de radio
\end{itemize}

\paragraph{4. Pantalla de Detalles de
Zona}\label{pantalla-de-detalles-de-zona}

\begin{itemize}
\tightlist
\item
  Cuadrícula de espacios
\item
  Estados visuales (disponible/ocupado/reservado)
\item
  Información de la zona
\item
  Botón de reserva
\end{itemize}

\paragraph{5. Pantalla de Reserva
Activa}\label{pantalla-de-reserva-activa}

\begin{itemize}
\tightlist
\item
  Temporizador countdown
\item
  Información del espacio
\item
  Botón de cancelación
\item
  Indicaciones para llegar
\end{itemize}

\paragraph{6. Pantalla de Perfil}\label{pantalla-de-perfil}

\begin{itemize}
\tightlist
\item
  Información del usuario
\item
  Estadísticas de uso
\item
  Lista de vehículos
\item
  Configuración
\end{itemize}

\paragraph{7. Pantalla de Historial}\label{pantalla-de-historial}

\begin{itemize}
\tightlist
\item
  Lista de reservas pasadas
\item
  Filtros por fecha y estado
\item
  Detalles de cada reserva
\end{itemize}

\subsubsection{8.3 Flujos de Navegación}\label{flujos-de-navegaciuxf3n}

\begin{Shaded}
\begin{Highlighting}[]
\NormalTok{flowchart LR}
\NormalTok{    A[Login] {-}{-}\textgreater{} B[Home]}
\NormalTok{    B {-}{-}\textgreater{} C[Mapa]}
\NormalTok{    B {-}{-}\textgreater{} D[Zona]}
\NormalTok{    D {-}{-}\textgreater{} E[Reserva]}
\NormalTok{    B {-}{-}\textgreater{} F[Mis Reservas]}
\NormalTok{    B {-}{-}\textgreater{} G[Historial]}
\NormalTok{    B {-}{-}\textgreater{} H[Perfil]}
\NormalTok{    H {-}{-}\textgreater{} I[Vehículos]}
\end{Highlighting}
\end{Shaded}

\subsubsection{8.4 Paleta de Colores}\label{paleta-de-colores}

\begin{longtable}[]{@{}lll@{}}
\toprule\noalign{}
Color & Uso & Hex \\
\midrule\noalign{}
\endhead
\bottomrule\noalign{}
\endlastfoot
\textbf{Primary} & Botones principales, AppBar & \#1976D2 \\
\textbf{Secondary} & Acentos, FABs & \#FF6F00 \\
\textbf{Success} & Espacios disponibles & \#4CAF50 \\
\textbf{Warning} & Advertencias, reservados & \#FFC107 \\
\textbf{Error} & Errores, ocupados & \#F44336 \\
\textbf{Background} & Fondo principal & \#FAFAFA \\
\textbf{Surface} & Tarjetas, diálogos & \#FFFFFF \\
\end{longtable}

\begin{center}\rule{0.5\linewidth}{0.5pt}\end{center}

\subsection{9. Base de Datos}\label{base-de-datos}

\subsubsection{9.1 Modelo de Datos}\label{modelo-de-datos}

EstacionaUNSA utiliza \textbf{Cloud Firestore}, una base de datos NoSQL
en tiempo real.

\paragraph{Colecciones Principales}\label{colecciones-principales}

\begin{Shaded}
\begin{Highlighting}[]
\NormalTok{erDiagram}
\NormalTok{    USERS ||{-}{-}o\{ VEHICLES : has}
\NormalTok{    USERS ||{-}{-}o\{ RESERVATIONS : makes}
\NormalTok{    PARKING\_ZONES ||{-}{-}o\{ PARKING\_SPOTS : contains}
\NormalTok{    PARKING\_SPOTS ||{-}{-}o\{ RESERVATIONS : "reserved by"}
\NormalTok{    USERS ||{-}{-}o\{ ENTRY\_EXIT\_LOGS : generates}
\NormalTok{    USERS ||{-}{-}o\{ INCIDENTS : "involved in"}
    
\NormalTok{    USERS \{}
\NormalTok{        string uid PK}
\NormalTok{        string email}
\NormalTok{        string displayName}
\NormalTok{        string role}
\NormalTok{        object stats}
\NormalTok{        timestamp createdAt}
\NormalTok{    \}}
    
\NormalTok{    VEHICLES \{}
\NormalTok{        string plate PK}
\NormalTok{        string type}
\NormalTok{        string model}
\NormalTok{        string color}
\NormalTok{        string photoURL}
\NormalTok{    \}}
    
\NormalTok{    PARKING\_ZONES \{}
\NormalTok{        string zoneId PK}
\NormalTok{        string name}
\NormalTok{        object location}
\NormalTok{        object capacity}
\NormalTok{        object schedule}
\NormalTok{    \}}
    
\NormalTok{    PARKING\_SPOTS \{}
\NormalTok{        string spotId PK}
\NormalTok{        string zoneId FK}
\NormalTok{        string status}
\NormalTok{        string type}
\NormalTok{        object currentReservation}
\NormalTok{    \}}
    
\NormalTok{    RESERVATIONS \{}
\NormalTok{        string reservationId PK}
\NormalTok{        string userId FK}
\NormalTok{        string spotId FK}
\NormalTok{        string status}
\NormalTok{        object time}
\NormalTok{        object location}
\NormalTok{    \}}
\end{Highlighting}
\end{Shaded}

\subsubsection{9.2 Reglas de Seguridad}\label{reglas-de-seguridad}

Las reglas de Firestore garantizan que:

\begin{itemize}
\tightlist
\item
  Solo usuarios autenticados pueden leer datos
\item
  Los usuarios solo pueden modificar sus propios datos
\item
  Las reservas tienen validaciones de negocio
\item
  Los vigilantes tienen permisos especiales para logs
\item
  Los administradores tienen acceso completo
\end{itemize}

\textbf{Ejemplo de regla:}

\begin{Shaded}
\begin{Highlighting}[]
\NormalTok{match }\OperatorTok{/}\NormalTok{reservations}\OperatorTok{/}\NormalTok{\{reservationId\} \{}
\NormalTok{  allow }\DataTypeTok{create}\OperatorTok{:} \ControlFlowTok{if}\NormalTok{ request}\OperatorTok{.}\AttributeTok{auth} \OperatorTok{!=} \KeywordTok{null} 
    \OperatorTok{\&\&}\NormalTok{ request}\OperatorTok{.}\AttributeTok{resource}\OperatorTok{.}\AttributeTok{data}\OperatorTok{.}\AttributeTok{userId} \OperatorTok{==}\NormalTok{ request}\OperatorTok{.}\AttributeTok{auth}\OperatorTok{.}\AttributeTok{uid}
    \OperatorTok{\&\&} \OperatorTok{!}\FunctionTok{hasActiveReservation}\NormalTok{(request}\OperatorTok{.}\AttributeTok{auth}\OperatorTok{.}\AttributeTok{uid}\NormalTok{)}\OperatorTok{;}
\NormalTok{  allow }\DataTypeTok{read}\OperatorTok{:} \ControlFlowTok{if}\NormalTok{ request}\OperatorTok{.}\AttributeTok{auth} \OperatorTok{!=} \KeywordTok{null} 
    \OperatorTok{\&\&}\NormalTok{ resource}\OperatorTok{.}\AttributeTok{data}\OperatorTok{.}\AttributeTok{userId} \OperatorTok{==}\NormalTok{ request}\OperatorTok{.}\AttributeTok{auth}\OperatorTok{.}\AttributeTok{uid}\OperatorTok{;}
\NormalTok{\}}
\end{Highlighting}
\end{Shaded}

\subsubsection{9.3 Optimizaciones}\label{optimizaciones}

\begin{itemize}
\tightlist
\item
  \textbf{Índices compuestos} para queries frecuentes
\item
  \textbf{Paginación} en listas largas
\item
  \textbf{Caché local} para reducir lecturas
\item
  \textbf{Listeners selectivos} para actualizaciones en tiempo real
\item
  \textbf{Transacciones} para operaciones atómicas
\end{itemize}

\begin{center}\rule{0.5\linewidth}{0.5pt}\end{center}

\subsection{10. Testing y Calidad}\label{testing-y-calidad-1}

\subsubsection{10.1 Resumen del Plan de
Testing}\label{resumen-del-plan-de-testing}

El proyecto implementó un \textbf{plan de testing de 5 fases}
documentado en \texttt{documentacion/testing/}:

\paragraph{Fase 1: Análisis Estático
}\label{fase-1-anuxe1lisis-estuxe1tico}

\begin{itemize}
\tightlist
\item
  Herramienta: Dart Analyzer (Flutter)
\item
  Hallazgos iniciales: 123 issues
\item
  Correcciones aplicadas: 30 errores críticos
\item
  Resultado final: 93 issues (mejora del 24\%)
\end{itemize}

\paragraph{Fase 2: Plan de Pruebas }\label{fase-2-plan-de-pruebas}

\begin{itemize}
\tightlist
\item
  Alcance y objetivos definidos
\item
  Tipos de pruebas: unitarias, integración, componentes, funcionales,
  sistema, UAT
\item
  Roles asignados al equipo
\item
  Herramientas justificadas
\end{itemize}

\paragraph{Fase 3: Casos de Prueba }\label{fase-3-casos-de-prueba}

\begin{itemize}
\tightlist
\item
  22 casos de prueba diseñados
\item
  Formato estandarizado con ID, precondiciones, pasos, resultados
  esperados
\item
  Distribución: 5 unitarias, 3 integración, 5 componentes, 3
  funcionales, 3 sistema, 2 UAT
\end{itemize}

\paragraph{Fase 4: Ejecución y Evidencias
}\label{fase-4-ejecuciuxf3n-y-evidencias}

\begin{itemize}
\tightlist
\item
  Capturas de pantalla
\item
  Logs de ejecución
\item
  Videos demostrativos
\item
  Re-ejecución de pruebas corregidas
\end{itemize}

\paragraph{Fase 5: Defectos y Reporte
}\label{fase-5-defectos-y-reporte}

\begin{itemize}
\tightlist
\item
  Registro de defectos encontrados
\item
  Matriz de trazabilidad
\item
  Informe final consolidado
\end{itemize}

\subsubsection{10.2 Resultados de Pruebas}\label{resultados-de-pruebas}

\begin{longtable}[]{@{}lllll@{}}
\toprule\noalign{}
Tipo de Prueba & Total & Aprobadas & Fallidas & Tasa de Éxito \\
\midrule\noalign{}
\endhead
\bottomrule\noalign{}
\endlastfoot
Unitarias & 5 & 5 & 0 & 100\% \\
Integración & 3 & 3 & 0 & 100\% \\
Componentes UI & 5 & 5 & 0 & 100\% \\
Funcionales & 3 & 3 & 0 & 100\% \\
Sistema & 3 & 3 & 0 & 100\% \\
UAT & 2 & 2 & 0 & 100\% \\
\textbf{TOTAL} & \textbf{22} & \textbf{22} & \textbf{0} &
\textbf{100\%} \\
\end{longtable}

\subsubsection{10.3 Defectos Encontrados y
Corregidos}\label{defectos-encontrados-y-corregidos}

Durante el análisis estático y las pruebas se encontraron y corrigieron:

\textbf{Críticos (19):} - Dependencia faltante:
\texttt{flutter\_local\_notifications} - Errores de importación en
\texttt{messaging\_service.dart} - Variables no utilizadas

\textbf{Altos (1):} - Importación no utilizada en
\texttt{home\_screen.dart}

\textbf{Medios (3):} - Warnings de análisis estático - Code smells
menores

\textbf{Deprecaciones (6):} - Uso de \texttt{withOpacity} en lugar de
\texttt{Color.fromRGBO}

\begin{center}\rule{0.5\linewidth}{0.5pt}\end{center}

\subsection{11. Mantenimiento}\label{mantenimiento}

\subsubsection{11.1 Plan de Mantenimiento}\label{plan-de-mantenimiento}

\paragraph{Mantenimiento Correctivo}\label{mantenimiento-correctivo}

\textbf{Responsabilidades:} - Monitoreo de errores en producción -
Corrección de bugs reportados por usuarios - Actualización de
dependencias con vulnerabilidades

\textbf{Frecuencia:} Según necesidad (reactivo)

\paragraph{Mantenimiento Preventivo}\label{mantenimiento-preventivo}

\textbf{Actividades:} - Actualización mensual de dependencias de Flutter
- Revisión trimestral de reglas de seguridad de Firebase - Optimización
de queries de Firestore - Limpieza de datos obsoletos

\textbf{Frecuencia:} Mensual/Trimestral

\paragraph{Mantenimiento Evolutivo}\label{mantenimiento-evolutivo}

\textbf{Mejoras planificadas:} - Integración con sensores IoT -
Aplicación web para administradores - Sistema de pagos - Reservas
programadas - Integración con cámaras de seguridad

\textbf{Frecuencia:} Según roadmap

\subsubsection{11.2 Actualizaciones
Futuras}\label{actualizaciones-futuras}

\paragraph{Corto Plazo (3-6 meses)}\label{corto-plazo-3-6-meses}

\begin{itemize}
\tightlist
\item[$\square$]
  Versión iOS de la aplicación
\item[$\square$]
  Panel web para vigilantes
\item[$\square$]
  Reportes y estadísticas avanzadas
\item[$\square$]
  Notificaciones por email
\end{itemize}

\paragraph{Mediano Plazo (6-12 meses)}\label{mediano-plazo-6-12-meses}

\begin{itemize}
\tightlist
\item[$\square$]
  Integración con sistema académico UNSA
\item[$\square$]
  Reservas programadas (con anticipación)
\item[$\square$]
  Sistema de favoritos de zonas
\item[$\square$]
  Modo oscuro (dark mode)
\end{itemize}

\paragraph{Largo Plazo (12+ meses)}\label{largo-plazo-12-meses}

\begin{itemize}
\tightlist
\item[$\square$]
  Sensores IoT para detección automática
\item[$\square$]
  Integración con cámaras de seguridad
\item[$\square$]
  Sistema de pagos para visitantes
\item[$\square$]
  Análisis predictivo de ocupación
\end{itemize}

\subsubsection{11.3 Escalabilidad}\label{escalabilidad}

\textbf{Capacidad actual:} - 120 espacios en 3 zonas -
\textasciitilde500-1000 usuarios estimados - Firebase Spark Plan
(gratuito)

\textbf{Escalabilidad:} -  Arquitectura preparada para más zonas - 
Firestore escala automáticamente -  Código modular y extensible - 
Requiere migración a plan de pago con \textgreater50K lecturas/día

\subsubsection{11.4 Backup y
Recuperación}\label{backup-y-recuperaciuxf3n}

\textbf{Estrategia:} - Backups automáticos de Firestore (diarios) -
Exportación mensual de datos críticos - Versionado de código en GitHub -
Documentación de configuración de Firebase

\begin{center}\rule{0.5\linewidth}{0.5pt}\end{center}

\subsection{12. Lecciones Aprendidas}\label{lecciones-aprendidas}

\subsubsection{12.1 Desafíos Enfrentados}\label{desafuxedos-enfrentados}

\paragraph{1. Gestión de Estado en Tiempo
Real}\label{gestiuxf3n-de-estado-en-tiempo-real}

\textbf{Desafío:} Mantener sincronizados múltiples widgets con datos de
Firestore en tiempo real.

\textbf{Solución:} Implementación de Provider Pattern con streams de
Firestore, permitiendo actualizaciones automáticas sin refrescos
manuales.

\textbf{Aprendizaje:} La combinación de Provider + Firestore Streams es
poderosa pero requiere manejo cuidadoso de listeners para evitar memory
leaks.

\paragraph{2. Validación de Distancia}\label{validaciuxf3n-de-distancia}

\textbf{Desafío:} Calcular con precisión la distancia del usuario a las
zonas de estacionamiento para validar reservas.

\textbf{Solución:} Uso del paquete \texttt{geolocator} con cálculo de
distancia haversine y manejo de permisos de ubicación.

\textbf{Aprendizaje:} Los servicios de ubicación pueden ser imprecisos
en interiores; se implementó un margen de tolerancia.

\paragraph{3. Transacciones Atómicas}\label{transacciones-atuxf3micas}

\textbf{Desafío:} Garantizar que las reservas no generen condiciones de
carrera (dos usuarios reservando el mismo espacio).

\textbf{Solución:} Uso de transacciones de Firestore para operaciones
atómicas de verificación y creación.

\textbf{Aprendizaje:} Las transacciones son esenciales para operaciones
críticas, aunque añaden complejidad al código.

\paragraph{4. Manejo de Notificaciones}\label{manejo-de-notificaciones}

\textbf{Desafío:} Configurar Firebase Cloud Messaging para funcionar en
diferentes estados de la app (foreground, background, terminated).

\textbf{Solución:} Implementación de handlers específicos para cada
estado y uso de local notifications.

\textbf{Aprendizaje:} Las notificaciones push requieren configuración
detallada en Android (permisos, canales) y testing exhaustivo.

\paragraph{5. Testing de Código
Asíncrono}\label{testing-de-cuxf3digo-asuxedncrono}

\textbf{Desafío:} Escribir tests para código que depende de Firebase y
operaciones asíncronas.

\textbf{Solución:} Uso de mocks y fakes para simular servicios de
Firebase en tests unitarios.

\textbf{Aprendizaje:} El testing de código asíncrono requiere paciencia
y comprensión profunda de Futures y Streams en Dart.

\subsubsection{12.2 Soluciones
Implementadas}\label{soluciones-implementadas}

\paragraph{Arquitectura Limpia}\label{arquitectura-limpia}

La separación en capas (UI, Providers, Services, Data) facilitó: -
Testing independiente de cada capa - Cambios sin afectar otras partes
del código - Onboarding más rápido de nuevos desarrolladores

\paragraph{Documentación Continua}\label{documentaciuxf3n-continua}

Mantener documentación actualizada en \texttt{documentacion/} permitió:
- Referencia rápida durante el desarrollo - Facilitar la colaboración
del equipo - Base para este informe final

\paragraph{Code Reviews}\label{code-reviews}

Las revisiones de código mediante pull requests ayudaron a: - Detectar
errores tempranamente - Compartir conocimiento entre el equipo -
Mantener calidad y consistencia del código

\subsubsection{12.3 Conocimientos
Adquiridos}\label{conocimientos-adquiridos}

\paragraph{Técnicos}\label{tuxe9cnicos}

\begin{itemize}
\tightlist
\item
  \textbf{Flutter avanzado:} Gestión de estado, navegación, widgets
  personalizados
\item
  \textbf{Firebase:} Firestore, Authentication, Cloud Messaging, reglas
  de seguridad
\item
  \textbf{Arquitectura de software:} Clean Architecture, patrones de
  diseño
\item
  \textbf{Testing:} Estrategias de testing para aplicaciones móviles
\item
  \textbf{Git/GitHub:} Flujo de trabajo colaborativo con branches y PRs
\end{itemize}

\paragraph{Blandas}\label{blandas}

\begin{itemize}
\tightlist
\item
  \textbf{Trabajo en equipo:} Coordinación y comunicación efectiva
\item
  \textbf{Gestión de tiempo:} Priorización de tareas y cumplimiento de
  deadlines
\item
  \textbf{Resolución de problemas:} Debugging y búsqueda de soluciones
\item
  \textbf{Documentación:} Importancia de documentar decisiones y
  procesos
\end{itemize}

\subsubsection{12.4 Mejoras Futuras
Sugeridas}\label{mejoras-futuras-sugeridas}

\begin{enumerate}
\def\labelenumi{\arabic{enumi}.}
\tightlist
\item
  \textbf{Implementar CI/CD:} Automatizar testing y deployment
\item
  \textbf{Añadir más tests:} Aumentar cobertura de código
\item
  \textbf{Mejorar UX:} Animaciones y transiciones más fluidas
\item
  \textbf{Optimizar rendimiento:} Reducir tiempo de carga inicial
\item
  \textbf{Accesibilidad:} Mejorar soporte para lectores de pantalla
\item
  \textbf{Internacionalización:} Soporte para múltiples idiomas
\end{enumerate}

\begin{center}\rule{0.5\linewidth}{0.5pt}\end{center}

\subsection{13. Conclusiones}\label{conclusiones}

\subsubsection{13.1 Logros del Proyecto}\label{logros-del-proyecto}

EstacionaUNSA ha cumplido exitosamente con sus objetivos, entregando una
solución funcional y escalable para la gestión de estacionamientos
universitarios. Los principales logros incluyen:

 \textbf{Aplicación funcional} con todas las características
planificadas\\
 \textbf{Arquitectura sólida} que facilita mantenimiento y extensión\\
 \textbf{Base de datos en tiempo real} con sincronización
instantánea\\
 \textbf{Sistema de testing completo} con alta cobertura\\
 \textbf{Documentación exhaustiva} técnica y de usuario\\
 \textbf{Experiencia de usuario intuitiva} basada en Material Design 3

\subsubsection{13.2 Impacto Esperado}\label{impacto-esperado}

La implementación de EstacionaUNSA en la UNSA puede generar:

\begin{itemize}
\tightlist
\item
  \textbf{Ahorro de tiempo:} Reducción de 10-15 minutos en búsqueda de
  estacionamiento
\item
  \textbf{Reducción de congestión:} Menor tráfico vehicular en horas
  pico
\item
  \textbf{Mejor experiencia:} Mayor satisfacción de la comunidad
  universitaria
\item
  \textbf{Control mejorado:} Datos para toma de decisiones sobre
  infraestructura
\item
  \textbf{Uso eficiente:} Optimización de espacios disponibles
\end{itemize}

\subsubsection{13.3 Viabilidad de
Implementación}\label{viabilidad-de-implementaciuxf3n}

El proyecto es viable para implementación real considerando:

\begin{itemize}
\tightlist
\item
  \textbf{Costo bajo:} Firebase ofrece plan gratuito generoso
\item
  \textbf{Mantenimiento simple:} No requiere servidores propios
\item
  \textbf{Escalabilidad:} Puede crecer con la demanda
\item
  \textbf{Tecnología probada:} Flutter y Firebase son tecnologías
  maduras
\item
  \textbf{Soporte disponible:} Equipo capacitado para mantenimiento
\end{itemize}

\subsubsection{13.4 Reflexión Final}\label{reflexiuxf3n-final}

El desarrollo de EstacionaUNSA ha sido una experiencia enriquecedora que
nos permitió aplicar conocimientos teóricos en un proyecto real con
impacto potencial en nuestra comunidad universitaria.

Hemos aprendido que el desarrollo de software va más allá de escribir
código: requiere planificación, diseño cuidadoso, testing riguroso,
documentación clara y trabajo en equipo efectivo.

Este proyecto sienta las bases para futuras mejoras y extensiones, y
demuestra que con las herramientas y metodologías adecuadas, es posible
crear soluciones tecnológicas de calidad que resuelvan problemas reales.

\begin{center}\rule{0.5\linewidth}{0.5pt}\end{center}

\subsection{14. Referencias}\label{referencias}

\subsubsection{Documentación Técnica}\label{documentaciuxf3n-tuxe9cnica}

\begin{itemize}
\tightlist
\item
  \href{https://docs.flutter.dev/}{Flutter Documentation}
\item
  \href{https://dart.dev/guides/language/language-tour}{Dart Language
  Tour}
\item
  \href{https://firebase.flutter.dev/}{Firebase for Flutter}
\item
  \href{https://m3.material.io/}{Material Design 3}
\item
  \href{https://pub.dev/packages/provider}{Provider Package}
\end{itemize}

\subsubsection{Recursos Utilizados}\label{recursos-utilizados}

\begin{itemize}
\tightlist
\item
  \href{https://pub.dev/packages/google_maps_flutter}{Google Maps
  Flutter Plugin}
\item
  \href{https://pub.dev/packages/geolocator}{Geolocator Package}
\item
  \href{https://firebase.google.com/docs/cloud-messaging}{Firebase Cloud
  Messaging}
\item
  \href{https://firebase.google.com/docs/firestore}{Cloud Firestore}
\end{itemize}

\subsubsection{Repositorio del Proyecto}\label{repositorio-del-proyecto}

\begin{itemize}
\tightlist
\item
  \textbf{GitHub:} https://github.com/Choflis/EstacionaUNSA.git
\end{itemize}

\end{document}
